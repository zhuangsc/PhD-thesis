\subsection{Power Prediction Models} Power prediction models have been extensively studied
over the years.  In particular, models based on PMC have been very successful in
predicting power
consumption~\cite{Bircher:2012:CSP:2196827.2196987,Bertran:2012:SEC:2457472.2457499,Bertran:2010:DRP:1810085.1810108,Goel:2010:PSP:1909624.1909734,Isci:2003:RPM:956417.956567}.
Our PMC-based model is based on the work of Bertran et
al.~\cite{Bertran:2012:SEC:2457472.2457499,Bertran:2010:DRP:1810085.1810108}, which aims
at providing insight into the way individual architectural components influence power
consumption.  Our PMC-based model extends Bertran's model to account for manufacturing
variability in terms of power consumption.  The original model requires carefully crafting
micro-benchmarks that isolate activity per architectural component, in order to train it.
We show that comparable results can be obtained by training the model with a small set of
kernel applications and a microbenchmark that stresses the memory unit.
\par
\subsection{Power-Aware Budgeting and Scheduling} Managing power has become an important
issue in HPC.  Etinski et al.~\cite{5999809} propose a job scheduling policy that seeks
the optimal frequency for parallel jobs in order to lower power consumption.  Sarood et
al.~\cite{Sarood:2014:MTO:2683593.2683682} use performance modeling to increase job
throughput in power constrained systems.  A power management of overprovisioned systems
has also been studied by Patki et al.~\cite{patki:2013:eho:2464996.2465009,7515666}.
Unlike our work, all sockets are viewed as homogeneous in terms of power consumption,
which can lead to suboptimal scheduling decisions.
\par
More recent work identifies the need to consider manufacturing variability when making
scheduling decisions or managing a system's power budget
~\cite{Chasapis:2016:RMM:2925426.2926279,Inadomi:2015:AMI:2807591.2807638,Gholkar:2016:PTH:2967938.2967961,Ellsworth:2015:DPS:2807591.2807643,Bailey:2015:FLP:2807591.2807637,Teodorescu:2008:VAS:1381306.1382152,Totoni:tech:2014}.
Inadomi et al.~\cite{Inadomi:2015:AMI:2807591.2807638} extensively study the impact of
manufacturing variability on a number of production clusters and propose a variation-aware
power budgeting framework.  They introduce variability to their prediction model in
similar fashion to our \textit{PR} model, by measuring power variability using a single
microbenchmark on each socket and then apply it to their original, variability agnostic,
prediction model.    In contrast to our work, their prediction model is used to guide work
balancing within an MPI application and not system wide job scheduling.  Unlike our
\textit{VT} model, Inadomi's and our \textit{PR} models assume that variability is
application independent, which is not correct in general.  As the effects of manufacturing
variability are expected to increase in future CPUs
\cite{Marathe:2017:ESP:3149412.3149421}, a more robust model is needed, such as our
\textit{VT} model.  Chasapis et al.~\cite{Chasapis:2016:RMM:2925426.2926279} and Totoni et
al.~\cite{Totoni:tech:2014} also employ runtime scheduling solutions for mitigating
manufacturing variability at application level.  Teodorescu et
al.~\cite{Teodorescu:2008:VAS:1381306.1382152} study the impact of manufacturing
variability and propose a linear programming algorithm to find the best parameters for
power budgeting with DVFS, instead of optimizing job scheduling.  Ellsworth et
al.~\cite{Ellsworth:2015:DPS:2807591.2807643} propose a power distribution framework that
optimizes an HPC cluster's power consumption under a certain system wide power budget.  In
contrast to our work, jobs are scheduled without considering power consumption, but power
is redistributed, favoring more power intensive jobs.  A two level solution for
overprovisioned clusters is presented by Gholkar et
al.~\cite{Gholkar:2016:PTH:2967938.2967961}, where a job scheduler is used at system level
to allocate nodes and distribute power.  The job scheduler predicts the total energy
consumption in order to make a scheduling decision, but unlike our work, the prediction
model does not consider variability.  Individual sockets may run under power constrains
and a second runtime scheduler decides the optimal configuration of active
processors and the power distribution among them in order to mitigate the power
variability.
\par
Our job scheduling policies can also be coupled with energy saving runtimes, to further
reduce power consumption.  
%However, any technique that throttles frequency or power, such
%as VDD, will expose the performance variability of the sockets.  Our models would need to
%be extended to consider different frequencies during training.   
For example, our approach can be coupled with
Adagio~\cite{rountree2009}, which detects the
critical path of MPI codes and uses DFVS to reduce power consumption of
non-critical pieces of work.
%hiding performance variability from the scheduler.

