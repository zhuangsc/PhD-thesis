The Adaptive Weight Precision (AWP) algorithm relies on the tolerance of DNNs to data representation formats smaller than the 32-bit Floating Point standard.
Indeed, previous work indicates that,
unlike scientific codes focused on solving partial differential equations or large linear systems,
neural networks do not always require 32-bit representation during training~\cite{bottou08, gupta15}. 
Even more, adding stochastic noise to certain variables during the learning phase
improves DNNs accuracy~\cite{murray94, bishop95, aud13}.
Nevertheless, when facing unknown scenarios in terms of new workloads or parameter settings,
the data representation requirements of DNNs 
are non-trivial to be determined and, to make things more complicated, they may change as the training phase progresses.

\begin{algorithm}%[H]%algorithm* occupies full page
\caption{Adaptive Weight Precision (AWP) Algorithm}
\label{alg:norm}
{\fontsize{9}{9}\selectfont
\begin{algorithmic}[1]
    \State BitsPerLayer := [$B_0, B_1, \hdots, B_{NumLayers}$]
    \Comment List storing the number of bits corresponding to the data representation of each layer
    \State IntervalCounter := [0, 0, $\hdots$, 0]
    \Comment List storing the number of times the relative change rate
             fails to meet the threshold per layer
    \For {batch := 0 \ldots NumBatches}
        \State Apply backpropagation to batch
        \For {layer := 0 \ldots NumLayers}
            %\State $\text{norm}_{batch, layer} := |\text{W}_{batch, layer}|$
            \State $\delta := \frac{(|\text{W}_{batch, layer}| - |\text{W}_{batch-1, layer}|)}{|\text{W}_{batch-1, layer}|}$
            \If {$\delta$ < T}
                \State $\text{IntervalCounter}_{layer}$ +=  1
            \EndIf
            \If {$\text{IntervalCounter}_{layer}$ == INTERVAL}
                \State $\text{BitsPerLayer}_{layer}$ += N
                %\Comment N is the number of bits to increase
                \State $\text{IntervalCounter}_{layer}$ := 0
            \EndIf
        \EndFor
        \EndFor
\end{algorithmic}
}
\end{algorithm}

The AWP algorithm dynamically  
determines data representation requirements per each network layer by monitoring the 
evolution of the $l^2$-norm of the weights.
AWP identifies the number of bits that are needed to represent 
DNNs weights and guarantees the progress of the training process.
AWP assigns the same data representation format to all weights 
belonging to a certain network layer.
The training starts with a relatively small data representation that is independently 
increased for each layer. 
%In other words, the adjustment of the data format 
%monotonically increases the number of bits to store weights.
%In our context, for the set of $n$ weights $\{ w_1 \hdots w_n \}$ belonging to a certain layer, 
%the $l^2$-norm is used in the following way:
%
%\begin{equation*}
%    |W|=\sqrt{\sum_{k=1}^{n} |w_k|^2}, 
%    \quad \mathrm{whereas} \quad 
%    W = 
%    \begin{pmatrix}
%        w_1 \\
%        w_2 \\
%        \vdots \\
%        w_n
%    \end{pmatrix}, \quad k=1,...,n.
%\end{equation*}

Algorithm~\ref{alg:norm} displays a pseudo-code description of AWP.
Once the backpropagation process has been applied to a given batch, AWP iterates over all network layers.
%Per each layer, AWp computes the change rate~$\delta$ and compares it with a threshold~$T$.
%To do so, 
The algorithm computes per each batch and network layer the $l^2$-norm of all its weights' values 
and derives the relative change rate~$\delta$ of the $l^2$-norm 
with regard to the previously processed batch. 
For the batch $i$, the change rate is defined as  
$\delta_i=(|W_i| - |W_{i-1}|)/|W_{i-1}|$, where $W_i$ is the vector of weights 
of a certain layer while batch $i$ is processed. 
Every time the change rate is below a given threshold \textit{T} for a certain 
layer, the algorithm accounts for it by increasing the $IntervalCounter$ parameter.
The algorithm increases $N$~bits of precision if the change 
rate is below \textit{T} during a certain number of batches defined by the parameter
\textit{INTERVAL} 
and sets the $IntervalCounter$ parameter of the corresponding layer to zero.
Section~\ref{sec:input_evaluation} describes how we determine the values of parameters \textit{T}, \textit{INTERVAL}, and \textit{N}.
%Once this counter reaches a certain value \textit{INTERVAL} for a certain layer, the algorithm 
%decides to increase the number of bits representing this layer's weights and 
%sets the $IntervalCounter$ variable of the corresponding layer to zero.
