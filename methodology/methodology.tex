
\chapter{Experimental Setup}
\label{chap:methodology}

In this Chapter we describe the hardware and software platforms used for the experimental
evaluation of this work.  In this work we assume that we work with parallel workloads that
are implemented with a programming model that allows dynamically controlling its
concurrency level.  Applications should also be able to be decomposed into concurrent
tasks.  The underlying hardware platform should offer the ability to impose user defined
power caps, at least per each socket (such as Intel CPU models that are equally or older
than the Sandy Bridge family of processors).  Moreover, older CPU models do not
demonstrate notable manufacturing variability.
We also assume the presence of a workload manager to manage running applications on an 
HPC cluster.   

\section{Hardware Platforms} 
\label{sec:platforms}
For our experimental evaluation we use two distinct HPC
clusters, the MareNostrum III supercomputer at Barcelona Supercomputing Center and the
Catalyst and Quartz clusters at Lawrence Livermore National Lab.  Marenostrum III is one
of Europe's largest supercomputers and represents the state-of-the-art in production
environments in HPC.  Our initial evaluation of the PARSECSs is conducted on MareNostrum
III, however, since it's a large production machine, access to specific MSR registers,
required for monitoring and capping power on Intel chips, is restricted.  For this reason
we also use the Catalyst and Quartz cluster, where MSRs are accessible by the user through
special kernel modules. 

\begin{itemize}
	\item \emph{MareNostrum III}: It consists of 3,056 compute nodes in total. Each node is
IBM System X server iDataPlex dx360 M4, composed of two 8-core Intel Sandy Bridge
processors E5-2.60Hz, 20MB of shared last-level cache. There are eight 4GB DDR3 DIMM's
running at 1.6GHz (a total of 32GB per node and 2GB per core). 

	\item \emph{Catalyst}: The Catalyst cluster~\cite{llnlconfluence} consists of 324 NUMA
nodes, each with two 12-core Intel Xeon E5-2695v2 sockets and equipped with 128GB of main
memory.  It can reach a peak performance of 149.3 PFLOPS.  Access to MSR counters is
granted to normal users through a kernel module.  We use 128 of these nodes (256 sockets)
in our experiments, which totals to 3,072 cores.

	\item \emph{Quartz}: The Quartz cluster~\cite{llnlconfluence} consists of 2,634 NUMA
nodes, each with two 16-core Intel Xeon E5-2695v4 sockets and equipped with 128GB of main
memory.  It can reach a peak performance of 3,251.4 PFLOPS.  For our experiments we use We
use 128 of these nodes (256 sockets) in our experiments (4,096 cores in total).  As with
the case of Catalyst, access to the RAPL interface is granted through a kernel module.
This is a larger production machine we use to gather application execution traces and later
use in our workload manager simulator.

\end{itemize}

\section{Software Stack}

\subsection{Runtime System}
We use the OpenMP 4.0 standard to implement the PARSECSs benchmark suite.  We make use of
the Nanos++ OpenMP runtime system (version 0.8a).  Because of it's modular design it is
ideal to expand it's functionality and offers all OpenMP 4.0 features along with some
experimental ones.  Note that we only use the standard OpenMP 4.0 features in our
implementation.  We also use Nano++ for developing our power-aware runtime approach.
Nanos++ is coupled with the Mercurium source-to-source compiler (version 1.99), and gcc
4.7 as the back-end compiler.  
%To analyze the behavior of the \PARSEC{} benchmark suite (version 3.0), we use the Extrae
%instrumentation package~\cite{Labarta2006} (version 2.5) and the Paraver trace
%viewer~\cite{Labarta2006} (version 4.5). 
%We run all benchmarks using their respective native inputs as described in Table~\ref{tab:parsec}. 

\subsection{Analysis tools and Power Capping Framework}
We use a number of tools to perform various types of analyses on our benchmarks.  We use
the Extrae instrumentation package~\cite{Labarta2006} (version 2.5) and the Paraver trace
viewer~\cite{Labarta2006} (version 4.5) to analyze and compare the PARSEC and PARSECSs
benchmark suites.
 

\textit{Performance and Power Monitoring} 
Our methodology requires to monitor performance counters plus power consumption rates.  We
use \textit{perf} version 3.10 and \textit{mpstat} version 10.5.1 for monitoring
architectural and core component activity.  For measuring power and enforcing power limits
we use Intel's RAPL registers, which expand typical hardware counters, offering precise
readings on power consumption and temperature, as well as offering the functionality to
constrain core power consumption to a certain limit.  On Sandy and Ivy Bridge CPUs, these
special registers are accessible per socket, but on newer architectures like Haskwell,
they offer the same features per core.  We implement a daemon built on top of
\textit{libmsr}~\cite{libmsr}, which is a user friendly framework for accessing RAPL
registers safely from user space, through a special kernel module.  Our study focuses on
the variability on processors.  In our experiments we measured variance in DRAM power
consumption of less than 1\% (measured using the RAPL interface), among different
sockets when running the same benchmark.  For this reason, we only report the power 
consumption of processors.  We use the same framework for power capping sockets on
Catalyst and Quartz.  These power caps are enforced at hardware level by reducing the 
effective frequency of all cores, to match the requested power budget.
The user can specify a time window and a maximum average power for that window.
The processor guarantees that it will not exceed this average.
Intuitively, longer windows may allow better performance
for applications that utilize the CPU in bursts; if the burst
exceeds the window size, the processor will have to be
throttled.

The sample rate is 100ms for power monitoring and 1s for performance counters.  Although
we are able to monitor an applications real power consumption on a finer grain, our
predictions are limited to 1s granularity, since they depend on the performance counter
data collected at the coarser granularity.

\section{Workload Manager Simulator}
\label{sec:simulator}
For testing our scheduling policies, we implement a discrete event simulator, and
implement our scheduling policies on top of it.  Although there already exist workload
manager simulators for SLURM \cite{slurm_sim} and Flux \cite{flux_sim}, they do not model
power.  Moreover, these simulators also require tracing the applications.  We found it to
be more practical to implement our own to better control what traces need to be generated
and how power is to be handled and modeled. 

The simulator  requires all jobs to be first executed on the physical hardware
to gather performance and power profiles,  which are then used to simulate
their execution under different scheduling schemes. The performance and power
profiles are referred to as traces in this document.  They track performance
and power information over time, throughout a job's execution.  These traces
are different to job queues' traces, which contain information on the time a
job is issued, scheduled and completed, as well as which resources were
allocated for its execution.  Idle power is modeled as the average power consumed by each
socket, as measured on the actual hardware.  Using a simulator allows us to
rapidly test and evaluate new policies without any accuracy loss since the
simulations are led by power traces obtained from real parallel executions.
Our design allows the user to implement scheduling policies using python
scripts.  In our setup we implement SLURM's \cite{slurm_02} default job
scheduling policy, since it's the de-facto resource management tool on
production clusters.  The additional policies suggested by this work expand the
functionality of SLURM's default scheduler with power- and
variability-awareness.

\subsubsection{Validation of Workload Manager Simulator}
To validate the simulator we run a small scale experiment using 8 nodes (16
sockets) on Quartz.  The workload we use consists of a mix of a 100 instances
of the PARSECSs benchmarks.  Each instance is a single socket job and is
randomly chosen out of the total 10 PARSECSs benchmarks.  We run a bash script
that periodically issues 10 instances (every 60s) until all the job instances
are issued.  Then, it waits for all jobs to finish execution.  We generate a
trace with the timestamp of when every job was issued and on which node it was
run on.  We also keep track of the total time it takes for all the jobs to
complete along with the total energy the sockets consumed (including idle time).        

We then use the simulator to repeat the experiment and reproduce the results measured on
the actual machine.  We use the same set of nodes and 100 instances.  
%The simulator requires application traces of the applications run on all sockets.  
We run all PARSECSs
applications on all 16 sockets to gather the traces and power profiles (on a different run 
from the original experiment described in the previous paragraph).  Then we issue the same
100 jobs again, this time on the simulator, at the same intervals as with the original run
on the actual machine.  The simulator will also use the workload trace to get the socket each
job run on, and try to replicate the same socket to job allocation, if possible.
This way, it tries to essentially recreating the same scheduling decisions SLURM took on the actual machine.
When all jobs finish execution, we measure the total execution time and energy
consumption.

Comparing total execution time and energy consumption between the two experiments shows
that the simulator is 1.6\% slower than the actual execution on the cluster.  Moreover,
the jobs' total power consumption is higher on the simulator by 1.1\%.  These results
show that our simulator has very good accuracy and the results discussed in Section 
\ref{chap:power_aware_job_scheduling} are representative of the impact our scheduling
policies would have on the actual machine.

\begin{table}
        \centering
        \caption{Benchmark training set for PMC-based power prediction model.}
        \label{tab:training_set}
				\def\arraystretch{1.5}%
				\begin{tabular}{ | c | m{12cm} | }
                \hline
                \textbf{Benchmark} & \textbf{Description} \\
                \hline
                \hline
                cholesky & cholesky factorization kernel \\
                \hline
                knn & K-nearest neighbours kernel \\
                \hline
                matmul & Floating point matrix multiplication kernel \\
                \hline
                md5 & MD5 message-digest algorithm \\
                \hline
                prk2\_stencil & Tests the efficiency with which a space-invariant symmetric filter (stencil) applies to images \\
                \hline
                qr\_tile & Tiled QR factorization kernel \\
                \hline
                sparseLU & Sparse LU factorization kernel\\
                \hline
                stap & Space-Time Adaptive Processing for radar detection of an objects position \\
                \hline
                symmatinv & Symmetric matrix inversion kernel \\
                \hline
                vector-redu & Computes the sum of the elements of a vector \\
                \hline
                mem\_bench & A micro-benchmark that stretches different memory levels \\
                \hline
        \end{tabular}
%	\vspace{-.5cm}
\end{table}


\section{Benchmark Applications}
\subsection{Prediction Model Training}
To train our models we use a set of small kernel and micro-benchmarks, listed in
Table~\ref{tab:training_set}, that capture different behaviors.  In addition to these
kernels, we design a microbenchmark that stresses each level of the memory hierarchy, in
order to measure the impact that each cache level has on power consumption.  Our set
consists of kernel applications which are representative HPC workloads, however often
larger and more exhaustive set of benchmarks are selected for training
\cite{Bertran:2012:SEC:2457472.2457499}.  Although larger sets can give better results, by
capturing a wider variety of application behaviors, we demonstrate that our set provides
comparable results, while it's easy to deploy.  In comparison, Bertran et. al
\cite{Bertran:2012:SEC:2457472.2457499} use a large set of 100 micro-benchmarks, fine
tuned to the underlying architecture.  This set would be ideal for our prediction models
as well, however it is not portable and requires significant effort and understanding of
the underlying architecture, which makes its deployment challenging.

\begin{table}
        \centering
        \caption{NAS Multi-Zone benchmarks are multinode applications that use both MPI and OpenMP
		to express parallelism.  MPI is used for inter-node communication and OpenMP is used 
		for intra-node parallelizations of loops.}
        \label{tab:nas_mz_set}
				\def\arraystretch{1.5}%
				\begin{tabular}{ | c | m{12cm} | }
                \hline
                \textbf{Benchmark} & \textbf{Description} \\
                \hline
                \hline
                BT-MZ & Block Tri-diagonal solver. The workload consists of a mesh, unevenly divided among processes. \\
                \hline
                LU-MZ & Lower-Upper Gauss-Seidel solver. The workload consists of a mesh, evenly divided among process. \\
                \hline
                SP-MZ & Scalar Penta-diagonal solver. The workload consists of a mesh, evenly divided among processes. \\
                \hline
        \end{tabular}
%	\vspace{-.5cm}
\end{table}



\subsection{Runtime and Job Scheduler Evaluation Benchmarks}
\label{sec:benchmarks}
To evaluate the runtime and job scheduling policies considered in this thesis, we use the
 PARSECSs benchmark suite, which we developed after the PARSEC benchmark suite (further
described in Section \ref{sec:parsec}).  The PARSECSs benchmark suite consists of emerging
workloads for shared memory architectures, representative of applications run on typical
HPC systems.  Our implementations use the OmpSs/OpenMP 4.0 programming model, which allows
us to use current and emerging realistic workloads under a sophisticated programming
environment.  This is essential for evaluating our runtime solution for mitigating
manufacturing variability and improving the energy efficiency of the programming model, as the
original PARSEC suite is implemented in Pthreads, and only a couple of them use OpenMP 3.0
constructs, such as parallel loops.  Using a model like OmpSs/OpenMP 4.0, allows us to
easily modify and test our runtime approach without the need to re-implement our methodology for every
benchmark using the Pthreads model.  Moreover, using tasks and dataflow relations allows
us to express more complex parallelization strategies, that are not possible to implement
with the typical fork-join model of Pthreads and OpenMP 3.0.  This allows us to evaluate
our proposal using applications that better represent contemporary parallel workloads.
We discuss our implementations of the PARSECSs in more detail in Chapter
\ref{chap:task_based_benchmarks}.  

In the case of the job scheduling techniques, we use the PARSECSs
as our set of single socket parallel jobs.  However,  multi-node jobs are also common in HPC
environment.  For this reason we expand our benchmark set with the
MPI+OpenMP versions of the NAS multi-zone benchmarks ~\cite{Jin:2006:PCM:1143496.1143503} (NAS-MZ), posing
as our multi-node jobs.  The NAS-MZ benchmarks are described in table \ref{tab:nas_mz_set}. 
%are also a well known set of kernel applications that can run on a large set of nodes, often used in HPC.  
The multi-node jobs
run with different configurations for 8, 16 and 64 MPI processes, where each process runs
on a single socket.  All instances and processes run on 16 cores, with the exception of
\textit{facesim} (8 cores), \textit{fluidanimate} (8 cores), \textit{lu-mz\_D.8} and
\textit{lu-mz\_D.8} (1 core per MPI process).  Our diverse set of applications can run
from a single core up to 768 cores, for the larger MPI codes.  We run all benchmarks 5
times and report median values.  This is done to minimize the impact of noise or unrelated
to manufacturing variability inference in our experiments.  However, note that variation
in performance and power observed in all of our experiments were below 2\% (when repeating 
the same run on the same socket).

%\subsection{The PARSEC benchmark suite}
%
\begin{table*}[!t]
	\centering
	\scriptsize
	\caption{\PARSEC{} Benchmark Suite}
	\begin{tabular}{|l|p{6cm}|p{3cm}|p{1cm}|}
	\hline
	\textbf{Benchmark} & \multicolumn{1}{|c|}{\textbf{Description}} & \multicolumn{1}{|c|}{\textbf{Native input}} & \multicolumn{1}{|c|}{\textbf{LOC}}\\
	\hline \hline
	blackscholes & Intel RMS benchmark. It calculates the prices for a portfolio of European options analytically with the Black-Scholes partial differential equation (PDE). & 10,000,000 options & 404 \\ \hline
	bodytrack & Computer vision application which tracks a 3D pose of a marker-less human body with multiple cameras through an image sequence. & 4 cameras, 261 frames, 4,000 particles, 5 annealing layers & 6,968\\ \hline
	canneal & Simulated cache-aware annealing to optimize routing cost of a chip design. & 2,500,000 elements, 6,000 temperature steps & 3,040\\ \hline
	dedup & Compresses a data stream with a combination of global compression and local compression in order to achieve high compression ratios. & 672 MB data & 3,401\\ \hline
	facesim & Intel RMS workload which takes a model of a human face and a time sequence of muscle activation and computes a visually realistic animation of the modeled face. & 100 frames, 372,126 tetrahedra & 34,134\\ \hline
	ferret & Content-based similarity search of feature-rich data such as audio, images, video, 3D shapes, etc. & 3,500 queries, 59,695 images database, find top 50 images & 10,552\\ \hline
	fluidanimate & Intel RMS application uses an extension of the Smoothed Particle Hydrodynamics (SPH) method to simulate an incompressible fluid for interactive animation purposes. & 500 frames, 500,000 particles & 2,348\\ \hline
	freqmine & Intel RMS application which employs an array-based version of the FP-growth (Frequent Pattern-growth) method for Frequent Itemset Mining (FIMI). & 250,000 HTML documents, minimum support 11,000 & 2,231\\ \hline
	raytrace & Intel RMS workload which renders an animated 3D scene. & 200 frames, 1,920$\times$1,080 pixels, 10 million polygons & 13,751\\ \hline
	streamcluster & Solves the online clustering problem. & 200,000 points per block, 5 block & 1,769\\ \hline
	swaptions & Intel RMS workload which uses the Heath-Jarrow-Morton (HJM) framework to price a portfolio of swaptions. & 128 swaptions, 1,000,000 simulations & 1,225\\ \hline
	vips & VASARI Image Processing System (VIPS), which includes fundamental image processing operations. & 18,000$\times$18,000 pixels & 127,957\\ \hline
	x264 & H.264/AVC (Advanced Video Coding) video encoder. & 512 frames, 1,920$\times$1,080 pixels & 29,329\\ \hline 
	\end{tabular}
	\label{tab:parsec}
	\vspace{1cm}
\end{table*}

With the prevalence of many-core processors and the increasing relevance of application domains that do not belong to the traditional HPC field, 
comes the need for programs 
representative of current and future parallel workloads. 
The \PARSEC{}~\cite{Bienia:PhD2011} features state-of-the art, 
computationally intensive algorithms and very diverse workloads from different areas of computing.
\PARSEC{} is comprised of 13 benchmark programs. 
The original suite makes use of the Pthreads parallelization model for all these benchmarks, 
except for \texttt{freqmine}, which is only available in OpenMP. 
The suite includes input sets for native machine execution, which are real input sets.
Table~\ref{tab:parsec} describes the different benchmarks included in the suite along with their respective native input and the lines of code 
(LOC) of each application.
We apply tasking parallelization strategies to 11 out of its 13 applications: \texttt{blackscholes}, \texttt{bodytrack}, \texttt{canneal}, \texttt{dedup}, \texttt{facesim}, \texttt{ferret}, 
\texttt{fluidanimate}, \texttt{freqmine}, \texttt{streamcluster} and \texttt{swaptions} and \texttt{x264}. 
We leave 2 applications out of this study: \texttt{raytrace} and \texttt{vips}.
\texttt{Vips} is a domain specific runtime system for image manipulation. 
Since vips is a runtime itself, it is not reasonable to implement it on top of another runtime system. 
Therefore we do not include this code in our evaluations. 
\texttt{Raytrace} code has the same extension as 
\texttt{ferret}, \texttt{facesim} and \texttt{bodytrack} and the same parallel model as \texttt{blackscholes}~\cite{Cook:2013:HEC:2508148.2485949}.
Therefore, since it does not offer any new insight, we do not consider the \texttt{Raytrace} code in this work.
 
We have a preliminary task-based implementation of the \texttt{x264} encoder, which scales up to 14x on a 16-core machine, the same as the Pthreads version.  
Since we just emulate the same parallel model as the original Pthreads version and obtain the same performance, we do not include this code in the results Section as it provides no insight.





\subsection{Job Scheduler Workload Generation}
\label{sec:cluster_traffic}
Typically workload manager schedulers are evaluated using workload traces from the job
queues of actual HPC clusters \cite{Etinski2012615,FEITELSON20142967}.  However, in our
case this is not applicable, since these type of traces do not contain information on
power and manufacturing variability.  Moreover, we are not able to create a job queue trace
out of the clusters we have access to, since reading performance counters such as the RAPL
interface requires root access.  This is not an option for us on these production
machines.  For these reasons we generate our own cluster workload combining single- and
multi-node applications, so that we can measure the performance and power profiles of the
workload.  A similar methodology is used in other power and manufacturing variability
related studies \cite{Patki:2015:PRM:2749246.2749262,Ellsworth:2015:DPS:2807591.2807643},
but in our approach we use a wider number and range of applications.  The applications
used are described in Section \ref{sec:benchmarks}.  

We generate two random job distributions as our workload on the cluster,
corresponding to bursty and heavy loads.  The bursty scenario consists of 763,
periodically creating a heavy load that requires a large number of sockets to
be served, even exceeding the systems total capacity, having jobs wait.
However, there are also time periods that the system may be idle or have only a
few jobs to serve. The heavy load scenario consists of 2286 jobs, where there
are always enough jobs to occupy the whole system, for 98\% of the total
execution (2\% corresponds to initial submission when the whole system is idle
and the few last jobs remaining at finalization, before all jobs complete and system returns to idle state).  
In the rest of this document, we use the term traffic when referring to the cluster's load. 

\subsection{Configuration Exploration Space}
Our runtime approach needs to try different configurations of power distribution and
number of active cores in order to find a favorable one.  Exhaustively exploring all
possible configurations is not feasible, so we describe how we choose our configuration
space.  We consider power bounds of 80W, 100W and 120W for total node power. If we allow a
power limit of 80W, we consider 5 different ways of distributing the power among the two
sockets of the NUMA node: 30W:50W, 35W:45W, 40W:40W, 45W:35W and 50W:30W as well as 36
ways of specifying the maximum concurrency allowed in each 2-socket NUMA node: 2-2, 4-2,
6-2, 8-2, 10-2, 12-2, 2-4, etc.  up to 12-12. In total, this leads to a total of 180
combinations.  Similarly, when allowing a power limit of 100W there are 8 ways of
distributing it, which combined with the 36 possible ways of distributing the concurrency,
leads us to a total of 324 combinations. Similarly, when the total power budget reaches
120W, the total number of combinations is 468. Overall, for each particular application we
have 972 different combinations.  Other Considerations: The results of these experiments
are machine dependent since each particular 12-core socket reacts in a different way when
a power limit is set.  Ideally, all 972 configurations per application should be executed
on many NUMA nodes to really account for many possible hardware reactions when a power
limit is set. However, due to the size of our experimental campaign, we randomly chose a
single 2-socket NUMA node for each considered application and run all 972 combinations on
it. Although this random choice can slightly influence the relative results between the
benchmarks, the general conclusions we extract from them remain unchanged.

\subsection{General Considerations}
Special care is needed when conducting our experiments in order to ensure that we minimize
interference not related to manufacturing variability (such as OS noise and NUMA effects).
To deal with such random effects, we run our benchmarks 5 times on each socket and observe 
the variability within the same node.  Our benchmarks demonstrate inter-node variability
is very low compared to the one observed when running the same application on
different nodes (inter-node variability is less than 2\%, while the intra-node one can be
over 15\%).  If in any case we observe higher than normal inter-node performance
variability, we discard the results and repeat the experiment.  A similar strategy is used
to deal with power consumption variability due to temperature variations.  We always
measure temperature and discard results that are not within the range of 38-42 $^\circ$C.
TurboBoost has also been disabled to make sure that the hardware mechanisms that can alter
the effective frequency of cores, other than throttling to maintain the power cap, 
do not interfere.  
%We are confident that variation relevant to the above that is captured
%in our experiments is significantly lower than manufacturing variability and does not harm
%our results. 

