% ******************************************************************************
% ****************************** Custom Margin *********************************

% Add `custommargin' in the document class options to use this section
% Set {innerside margin / outerside margin / topmargin / bottom margin}  and
% other page dimensions
\ifsetCustomMargin
  \RequirePackage[left=37mm,right=30mm,top=35mm,bottom=30mm]{geometry}
  \setFancyHdr % To apply fancy header after geometry package is loaded
\fi

% Add spaces between paragraphs
%\setlength{\parskip}{0.5em}
% Ragged bottom avoids extra whitespaces between paragraphs
\raggedbottom
% To remove the excess top spacing for enumeration, list and description
%\usepackage{enumitem}
%\setlist[enumerate,itemize,description]{topsep=0em}

% *****************************************************************************
% ******************* Fonts (like different typewriter fonts etc.)*************

% Add `customfont' in the document class option to use this section

\ifsetCustomFont
  % Set your custom font here and use `customfont' in options. Leave empty to
  % load computer modern font (default LaTeX font).
  %\RequirePackage{helvet}

  % For use with XeLaTeX
  %  \setmainfont[
  %    Path              = ./libertine/opentype/,
  %    Extension         = .otf,
  %    UprightFont = LinLibertine_R,
  %    BoldFont = LinLibertine_RZ, % Linux Libertine O Regular Semibold
  %    ItalicFont = LinLibertine_RI,
  %    BoldItalicFont = LinLibertine_RZI, % Linux Libertine O Regular Semibold Italic
  %  ]
  %  {libertine}
  %  % load font from system font
  %  \newfontfamily\libertinesystemfont{Linux Libertine O}
\fi

% *****************************************************************************
% **************************** Custom Packages ********************************

% ************************* Algorithms and Pseudocode **************************

%\usepackage{algpseudocode}


% ********************Captions and Hyperreferencing / URL **********************

% Captions: This makes captions of figures use a boldfaced small font.
%\RequirePackage[small,bf]{caption}

\RequirePackage[labelsep=space,tableposition=top]{caption}
\renewcommand{\figurename}{Fig.} %to support older versions of captions.sty


% *************************** Graphics and figures *****************************

%\usepackage{rotating}
%\usepackage{wrapfig}

% Uncomment the following two lines to force Latex to place the figure.
% Use [H] when including graphics. Note 'H' instead of 'h'
%\usepackage{float}
%\restylefloat{figure}

% Subcaption package is also available in the sty folder you can use that by
% uncommenting the following line
% This is for people stuck with older versions of texlive
%\usepackage{sty/caption/subcaption}
\usepackage{subcaption}

% ********************************** Tables ************************************
\usepackage{booktabs} % For professional looking tables
\usepackage{multirow}

%\usepackage{multicol}
%\usepackage{longtable}
%\usepackage{tabularx}


% *********************************** SI Units *********************************
\usepackage{siunitx} % use this package module for SI units


% ******************************* Line Spacing *********************************

% Choose linespacing as appropriate. Default is one-half line spacing as per the
% University guidelines

% \doublespacing
% \onehalfspacing
% \singlespacing


% ************************ Formatting / Footnote *******************************

% Don't break enumeration (etc.) across pages in an ugly manner (default 10000)
%\clubpenalty=500
%\widowpenalty=500

%\usepackage[perpage]{footmisc} %Range of footnote options


% *****************************************************************************
% *************************** Bibliography  and References ********************

%\usepackage{cleveref} %Referencing without need to explicitly state fig /table

% Add `custombib' in the document class option to use this section
\ifuseCustomBib
   \RequirePackage[square, sort, numbers]{natbib} % CustomBib

% If you would like to use biblatex for your reference management, as opposed to the default `natbibpackage` pass the option `custombib` in the document class. Comment out the previous line to make sure you don't load the natbib package. Uncomment the following lines and specify the location of references.bib file

%\RequirePackage[backend=biber, style=numeric-comp, citestyle=numeric, sorting=nty, natbib=true]{biblatex}
%\bibliography{References/references} %Location of references.bib only for biblatex

\fi

% changes the default name `Bibliography` -> `References'
\renewcommand{\bibname}{References}


% ******************************************************************************
% ************************* User Defined Commands ******************************
% ******************************************************************************

% *********** To change the name of Table of Contents / LOF and LOT ************

%\renewcommand{\contentsname}{My Table of Contents}
%\renewcommand{\listfigurename}{My List of Figures}
%\renewcommand{\listtablename}{My List of Tables}


% ********************** TOC depth and numbering depth *************************

\setcounter{secnumdepth}{2}
\setcounter{tocdepth}{2}


% ******************************* Nomenclature *********************************

% To change the name of the Nomenclature section, uncomment the following line

%\renewcommand{\nomname}{Symbols}


% ********************************* Appendix ***********************************

% The default value of both \appendixtocname and \appendixpagename is `Appendices'. These names can all be changed via:

%\renewcommand{\appendixtocname}{List of appendices}
%\renewcommand{\appendixname}{Appndx}

% *********************** Configure Draft Mode **********************************

% Uncomment to disable figures in `draft'
%\setkeys{Gin}{draft=true}  % set draft to false to enable figures in `draft'

% These options are active only during the draft mode
% Default text is "Draft"
%\SetDraftText{DRAFT}

% Default Watermark location is top. Location (top/bottom)
%\SetDraftWMPosition{bottom}

% Draft Version - default is v1.0
%\SetDraftVersion{v1.1}

% Draft Text grayscale value (should be between 0-black and 1-white)
% Default value is 0.75
%\SetDraftGrayScale{0.8}


% ******************************** Todo Notes **********************************
%% Uncomment the following lines to have todonotes.

%\ifsetDraft
%	\usepackage[colorinlistoftodos]{todonotes}
%	\newcommand{\mynote}[1]{\todo[author=kks32,size=\small,inline,color=green!40]{#1}}
%\else
%	\newcommand{\mynote}[1]{}
%	\newcommand{\listoftodos}{}
%\fi

% Example todo: \mynote{Hey! I have a note}

\usepackage{epigraph}
\usepackage{savesym}
\usepackage{amsmath}
\usepackage{array}
\usepackage{listings}
\usepackage{epsfig}
\usepackage{float}
\usepackage{subcaption}
\captionsetup{compatibility=false}
\usepackage{epstopdf}
\usepackage{enumitem}
\usepackage{graphicx}
\usepackage{url}
\usepackage{amssymb}
\usepackage{pifont}% http://ctan.org/pkg/pifont
%\usepackage{acro}
\savesymbol{printindex}
\usepackage[acronym, toc]{glossaries}
\restoresymbol{GLOS}{printindex}
%\usepackage{biblatex}
%\usepackage[square,numbers]{natbib}
%\usepackage{cite}
%\usepackage[sorting=none]{biblatex}
\usepackage{bibentry}
%\usepackage{netbib}
\nobibliography*
%\usepackage[citestyle=verbose]{biblatex} 
\usepackage{pdfpages}
\usepackage{algorithm}
\usepackage{algpseudocode}
\usepackage{xcolor}
\usepackage{bytefield}
\usepackage{mathtools}
\usepackage{float}
\usepackage{setspace}
\usepackage{textcomp}
\usepackage{caption}
\usepackage{multirow}
\usepackage{soul}
\usepackage{balance}
\usepackage{array}
\usepackage{xr}
\usepackage[bottom]{footmisc}
\newcolumntype{P}[1]{>{\centering\arraybackslash}p{#1}}

\lstset{ %
  basicstyle=\scriptsize,        % the size of the fonts that are used for the code
  breakatwhitespace=false,         % sets if automatic breaks should only happen at whitespace
  breaklines=true,                 % sets automatic line breaking
  %commentstyle=\color{mygreen},    % comment style
  %escapeinside={\%*}{*)},          % if you want to add LaTeX within your code
  %frame=single,                    % adds a frame around the code
  keepspaces=true,                 % keeps spaces in text, useful for keeping indentation of code (possibly needs columns=flexible)
  keywordstyle=\color{blue},       % keyword style
  language=C,                 % the language of the code
  morekeywords={pragma, omp, task, in, out, taskwait},            % if you want to add more keywords to the set
  numbers=none,                    % where to put the line-numbers; possible values are (none, left, right)
  %numbersep=5pt,                   % how far the line-numbers are from the code
  %numberstyle=\tiny\color{mygray}, % the style that is used for the line-numbers
  %rulecolor=\color{black},         % if not set, the frame-color may be changed on line-breaks within not-black text (e.g. comments (green here))
  showspaces=false,                % show spaces everywhere adding particular underscores; it overrides 'showstringspaces'
  showstringspaces=false,          % underline spaces within strings only
  showtabs=false,                  % show tabs within strings adding particular underscores
  %stepnumber=2,                    % the step between two line-numbers. If it's 1, each line will be numbered
  %stringstyle=\color{mymauve},     % string literal style
  tabsize=2,                       % sets default tabsize to 2 spaces
  %title=\lstname                   % show the filename of files included with \lstinputlisting; also try caption instead of title
}

\usepackage{hyperref}
\hypersetup{
    colorlinks,
    citecolor=black,
    filecolor=black,
    linkcolor=black,
    urlcolor=blue
}

\renewcommand*\listfigurename{List of Figures}
\renewcommand*\listtablename{List of Tables}
\renewcommand{\bibname}{Bibliography}

\newcommand{\comment}{\textcolor{red}{}\textcolor{red}}

\newcommand{\cmark}{\ding{51}}%
\newcommand{\xmark}{\ding{55}}%

\newcommand{\OMPSS}{OmpSs}
\newcommand{\PARSEC}{PARSEC}

\newcommand{\ARCH}{Broadwell}

\newcommand{\DefaultSched}{\textit{SLURM extended}}
\newcommand{\PESched}{\textit{SLURM+PE}}
\newcommand{\PEVASched}{\textit{SLURM+PEVA}}
\newcommand{\PRVSSched}{\textit{SLURM+PRVP}}
\newcommand{\PMCVSSched}{\textit{SLURM+PMCVP}}
\newcommand{\IVSSched}{\textit{SLURM+IVP}}
\newcommand{\AvgJTT}{\textcolor{black}{24}}
\newcommand{\MaxJTT}{\textcolor{black}{30}}
\newcommand{\AvgEnergy}{\textcolor{black}{4}}
\newcommand{\MaxEnergy}{\textcolor{black}{8}}
\newcommand{\EstMaxJTT}{\textcolor{black}{12}}
\newcommand{\EstMinJTT}{\textcolor{black}{33}}

\renewcommand{\glsnamefont}[1]{\textbf{#1}}
\makeglossaries
 
\newacronym{hpc}{HPC}{High Performance Computing}
\newacronym{mpi}{MPI}{Message Passing Interface}
\newacronym{api}{API}{Application Programming Interface}
\newacronym{cg}{CG}{Conjugate Gradient Method}
\newacronym{minres}{MINRES}{Minimal Residual Method}
\newacronym{gmres}{GMRES}{Generalized Minimal Residual Method}
\newacronym{bicg}{BiCG}{Biconjugate Gradient Method}
\newacronym{dnn}{DNN}{Deep Neural Network}
\newacronym{cnn}{CNN}{Convolutional Neural Network}
\newacronym{nlp}{NLP}{Natural Language Processing}
\newacronym{rnn}{RNN}{Recurrent Neural Network}
\newacronym{mlp}{MLP}{Multi-Layer Perceptron}
\newacronym{cpu}{CPU}{Central Processing Unit}
\newacronym{gpu}{GPU}{Graphics Processing Unit}
\newacronym{fpga}{FPGA}{Field Programmable Gate Array}
\newacronym{asic}{ASIC}{Application Specific Integrated Circuit}
\newacronym{vlsi}{VLSI}{Very Large Scale Integration}
\newacronym{tdg}{TDG}{Task Dependency Graph}
\newacronym{cuda}{CUDA}{Compute Unified Device Architecture}

\makeatletter
\renewcommand{\@chapapp}{}% Not necessary...
\newenvironment{chapquote}[2][2em]
  {\setlength{\@tempdima}{#1}%
   \def\chapquote@author{#2}%
   \parshape 1 \@tempdima \dimexpr\textwidth-2\@tempdima\relax%
   \itshape}
  {\par\normalfont\hfill--\ \chapquote@author\hspace*{\@tempdima}\par\bigskip}
\makeatother

\setlength \epigraphwidth {\textwidth}
\setlength \epigraphrule {0pt}
%\AtBeginDocument{\renewcommand {\epigraphflush}{center}}
%\renewcommand {\sourceflush} {center}

