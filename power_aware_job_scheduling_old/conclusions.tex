In this Chapter we demonstrate that taking into account manufacturing variability to drive 
job scheduling policies provides significant benefits in terms of performance and power 
consumption. We propose two job scheduling policies, each one using a different power 
prediction model: the first assumes that power variability impacts all application equally, 
while the second one aims at obtaining the power variability impact per application using a 
PMC-based model. We compare both approaches with a range of state-of-the-art approaches as 
well as an approach using an oracle model. 
\par
We examine the benefits of our policies under bursty and heavy traffic scenarios and different power budgets.  We observe significant improvements on job turnaround time 
(up to \MaxJTT\% and \AvgJTT\% on average) and energy consumption (reducing it up to 
\MaxEnergy\% and \AvgEnergy\% on average) when compared to state-of-the-art approaches that 
do not consider appropriately the manufacturing variability in existing processors.
Moreover, the model-driven policies proposed in this work accurately predict the variability 
per socket and, as a result, they guarantee that the power consumption always remains below 
the system-wide budget, while the policies that rely on user estimations or prior power 
profiling fail to do so.
   
