
\begin{table*}[!t]
	\centering
	\scriptsize
	\caption{\PARSEC{} Benchmark Suite}
	\begin{tabular}{|l|p{6cm}|p{3cm}|p{1cm}|}
	\hline
	\textbf{Benchmark} & \multicolumn{1}{|c|}{\textbf{Description}} & \multicolumn{1}{|c|}{\textbf{Native input}} & \multicolumn{1}{|c|}{\textbf{LOC}}\\
	\hline \hline
	blackscholes & Intel RMS benchmark. It calculates the prices for a portfolio of European options analytically with the Black-Scholes partial differential equation (PDE). & 10,000,000 options & 404 \\ \hline
	bodytrack & Computer vision application which tracks a 3D pose of a marker-less human body with multiple cameras through an image sequence. & 4 cameras, 261 frames, 4,000 particles, 5 annealing layers & 6,968\\ \hline
	canneal & Simulated cache-aware annealing to optimize routing cost of a chip design. & 2,500,000 elements, 6,000 temperature steps & 3,040\\ \hline
	dedup & Compresses a data stream with a combination of global compression and local compression in order to achieve high compression ratios. & 672 MB data & 3,401\\ \hline
	facesim & Intel RMS workload which takes a model of a human face and a time sequence of muscle activation and computes a visually realistic animation of the modeled face. & 100 frames, 372,126 tetrahedra & 34,134\\ \hline
	ferret & Content-based similarity search of feature-rich data such as audio, images, video, 3D shapes, etc. & 3,500 queries, 59,695 images database, find top 50 images & 10,552\\ \hline
	fluidanimate & Intel RMS application uses an extension of the Smoothed Particle Hydrodynamics (SPH) method to simulate an incompressible fluid for interactive animation purposes. & 500 frames, 500,000 particles & 2,348\\ \hline
	freqmine & Intel RMS application which employs an array-based version of the FP-growth (Frequent Pattern-growth) method for Frequent Itemset Mining (FIMI). & 250,000 HTML documents, minimum support 11,000 & 2,231\\ \hline
	raytrace & Intel RMS workload which renders an animated 3D scene. & 200 frames, 1,920$\times$1,080 pixels, 10 million polygons & 13,751\\ \hline
	streamcluster & Solves the online clustering problem. & 200,000 points per block, 5 block & 1,769\\ \hline
	swaptions & Intel RMS workload which uses the Heath-Jarrow-Morton (HJM) framework to price a portfolio of swaptions. & 128 swaptions, 1,000,000 simulations & 1,225\\ \hline
	vips & VASARI Image Processing System (VIPS), which includes fundamental image processing operations. & 18,000$\times$18,000 pixels & 127,957\\ \hline
	x264 & H.264/AVC (Advanced Video Coding) video encoder. & 512 frames, 1,920$\times$1,080 pixels & 29,329\\ \hline 
	\end{tabular}
	\label{tab:parsec}
	\vspace{1cm}
\end{table*}

With the prevalence of many-core processors and the increasing relevance of application domains that do not belong to the traditional HPC field, 
comes the need for programs 
representative of current and future parallel workloads. 
The \PARSEC{}~\cite{Bienia:PhD2011} features state-of-the art, 
computationally intensive algorithms and very diverse workloads from different areas of computing.
\PARSEC{} is comprised of 13 benchmark programs. 
The original suite makes use of the Pthreads parallelization model for all these benchmarks, 
except for \texttt{freqmine}, which is only available in OpenMP. 
The suite includes input sets for native machine execution, which are real input sets.
Table~\ref{tab:parsec} describes the different benchmarks included in the suite along with their respective native input and the lines of code 
(LOC) of each application.
We apply tasking parallelization strategies to 11 out of its 13 applications: \texttt{blackscholes}, \texttt{bodytrack}, \texttt{canneal}, \texttt{dedup}, \texttt{facesim}, \texttt{ferret}, 
\texttt{fluidanimate}, \texttt{freqmine}, \texttt{streamcluster} and \texttt{swaptions} and \texttt{x264}. 
We leave 2 applications out of this study: \texttt{raytrace} and \texttt{vips}.
\texttt{Vips} is a domain specific runtime system for image manipulation. 
Since vips is a runtime itself, it is not reasonable to implement it on top of another runtime system. 
Therefore we do not include this code in our evaluations. 
\texttt{Raytrace} code has the same extension as 
\texttt{ferret}, \texttt{facesim} and \texttt{bodytrack} and the same parallel model as \texttt{blackscholes}~\cite{Cook:2013:HEC:2508148.2485949}.
Therefore, since it does not offer any new insight, we do not consider the \texttt{Raytrace} code in this work.
 
We have a preliminary task-based implementation of the \texttt{x264} encoder, which scales up to 14x on a 16-core machine, the same as the Pthreads version.  
Since we just emulate the same parallel model as the original Pthreads version and obtain the same performance, we do not include this code in the results Section as it provides no insight.



