
In this Chapter we evaluate the benefits of task-based parallelisim by applying it to the \PARSEC{} benchmark suite.
We discuss and compare our implementations to their
PARSEC Pthreads/OpenMP counterparts. 
We show how task parallelism can be applied on a wide range of applications from 
different domains.   
In fact, by comparing the lines of code between our implementations and the original versions, we make a strong case
that task-based  models are actually easier to use.
The asynchronous nature of task-based parallelism, along with data dependency tracking through dataflow annotations, allows
us to overlap computation with I/O phases.
The underlying runtime system can take care of issues like scheduling and load balancing without significant overhead. 

Our experimental results demonstrate that the task model can be easily applied on a wide range of applications beyond the HPC domain. 
Although, not all applications can benefit from a task-based approach, there are cases where 
it can greatly improve scalability. 
The programs that benefit most are those that present pipeline execution model, where different stages of the application can run concurrently.
The proposed benchmark suite is expected to be of great use in evaluating experimental software
and hardware system designs, offering a more mature testbed compared to the typically used
small kernel applications.
