This chapter presents the \emph{Altsplit} approach that intends to provide a
model parallelism of the DNN with less communication. It achieves this by
distributing the neurons and replicating them across all the computational
units alternately in-between successive layers so that all-to-all
communication only occurs every other layers instead of in a lock-step
fashion in an \emph{state-of-the-art} approach. Our experiments are conducted taking
multiple hyper-parameters into consideration: the number of neurons per
layer, the number of layers and batch sizes. We conduct the experiments on
two high-end multicore multinode clusters with distinct CPU architecture.

We find that the \emph{Altsplit} approach achieves significant speedups over
our baseline \emph{state-of-the-art} approach regardless the underlying cluster.
Furthermore, we show that the speedup is retained among various
hyper-parameters and we visualize the speedup by generating traces on both
approach and shows that the trade-off between additional floating-point 
computations and reduced communication pays off.
