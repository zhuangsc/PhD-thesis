\chapter{Conclusions}
\label{chap:conclusions}
This thesis intends to alleviate the bottlenecks brought about by all-to-all 
communication in the modern HPC systems due to the constant scaling-up of the 
system size, problem size and the appearances of emerging fields.
This proves to be a daunting task because there is not an one-size-fits-all 
approach to it. Each field possesses its own requirements and patterns on 
communication. Furthermore, the trade-offs applicable differ from field to 
field.

This thesis utilizes three trade-off techniques in concrete for the communication 
reduction purpose.
\begin{itemize}
    \item Exploiting the resilience towards accumulation of rounding errors and 
        loss of precision of the problem at hand to reduce communication.
    \item Trading with a decreased computational precision with a marginal 
        deterioration of the accuracy of the problem for reduced amount of 
        communication. 
    \item Trading at the cost of additional computation with reduce amount of 
        communication.
\end{itemize}

The thesis first takes on reducing communication in one of the Krylov iterative 
methods, CG. Compare to the various available direct methods the CG method 
displays greater tolerance towards the accumulation of rounding errors. We thus 
fuse a certain amount of iterations together which means some efforts to bring 
down the accumulated error such as the residual replacement strategy has to be 
deferred to the end of the fused phase. Nevertheless, we show that further 
incorporating a task-based parallelism approach, significant speedup can be 
achieved without much hampering to the convergence of the algorithm.

On tackling the problem of accelerating the DNN training with multiple GPUs, we 
exploited the fact that DNNs are intrinsically tolerant to a lowered precision 
of both its parameters. Guided by the L2-norm of the weights of each layer, we 
start by transferring compressed weights with low precision and dynamically 
increment the precision in a per-layer granularity if needed.
Our experiments confirms that this approach greatly reduces the amount of data
needed to be transferred from the CPU host to GPUs prior to the beginning of 
each batch and consequently outperforms the training with full 32-bit floating 
point weights in terms of the training time.

In order to remove some amount of communication from applying model parallelism 
to DNNs, we try to replicate every other layers rather than splitting each and 
every layer so that communication only occurs once every two layers. Despite the 
higher count of computation inevitably introduced by the approach, our 
experiments indicate that it is a trade-off well justified. It outperforms our 
baseline approach by a large margin on two high-end clusters with completely 
distinct CPU architectures. 

\section{Further Down The Road}
With the first exascale clusters on the horizon, the pursuit of more accurate 
models in computational sciences and the awe-inspiring speed deep learning and 
data science are taking over multiple industries, the potential benefit for a 
communication-reduction solution to a particular problem will not diminish any 
time soon. The techniques this thesis explores all come at 

\begin{appendices}
\chapter{Publications}
\section{Publications Related With The Thesis}
\begin{itemize}
	\item \bibentry{ifcg}
	\item Sicong Zhuang, Cristiano Malossi and Marc Casas. Reducing Data Motion 
        to Accelerate the Training of Deep Neural Networks (submitted to IPDPS'20)
	\item Sicong Zhuang, Panagiotis Hadjidoukas, Cristiano Malossi and Marc 
        Casas. Altsplit: Communication Reduction In DNN Model Parallelism 
        (future submission)
\end{itemize}

\section{Other Publications}
\begin{itemize}
    \item \bibentry{ilia}
\end{itemize}

\chapter{Pragmas}
\section{Description}
The following are the list of the \textit{pragma} annotations of the routines 
used in the various CG implementations of Chapter~\ref{chap:ifcg}. The 
source code is open source and can be downloaded from
\href{git@github.com:zhuangsc/IFCG.git}{Github}. The annotations follow the 
syntax and semantics of the OmpSs programming model. The exact definitions 
of those annotations can be found at the
\href{https://pm.bsc.es/ftp/ompss/doc/spec/}{OmpSs official page}.

\section{Annotations}
\begin{lstlisting}
#pragma omp task in(X[initx:initx+bm-1], Y[inity:inity+bm-1]) concurrent(result[0:bn-1]) no_copy_deps priority(p) label(ddot) 
void __t_dot(int p, int bm, int bn, int m, int n, double *X, double *Y, int initx, int inity, double *result); 

#pragma omp task in(X1[0:bm-1], X2[0:bm-1], Anum[0:bn-1], Aden[0:bn-1], Y1[0:bm-1], Y2[0:bm-1]) out(Z1[0:bm-1], Z2[0:bm-1]) no_copy_deps priority(1) label(dcpaxpy_comb)
void __t_cpaxpy_comb(int bm, int bn, int m, int n, double alpha, double *Anum, double *Aden, double *X1, double *X2, double *Y1, double *Y2, double *Z1, double *Z2);

#pragma omp task in(X[0:bm-1], Y[0:bm-1], SAnum[0:bn-1], SAden[0:bn-1]) out(Z[0:bm-1]) no_copy_deps priority(p) label(extm_axpy)
void __t_extm_axpy(int bm, int bn, int m, int n, double *SAnum, double *SAden, double *X, double *Y, double *Z, int p);

#pragma omp task in(X[initx:initx+bm-1], Y[inity:inity+bm-1], A[inita:inita+bm-1], B[initb:initb+bm-1]) concurrent([bn]result, [bn]result2) no_copy_deps priority(p) label(cg_dot2)
void _cg_dot2(int p, int bm, int bn, int m, int n, double *X, double *Y, int initx, int inity, double *result, double *A, double *B, int inita, int initb, double *result2);

/* Computation of the coefficients */
#pragma omp task in(gamma[i], delta) out(beta[i], alpha[i]) label(centinel) 
{
    if ( k > 0 ) {
        beta[i] = gamma[i]/gamma[iprev];
        alpha[i] = gamma[i] / (delta - beta[i] * gamma[i] / alpha[iprev]);
    } else {
        beta[i] = (double) 0;
        alpha[i] = gamma[i]/delta;
    }
    gamma[iprev] = delta = 0;
}  
\end{lstlisting}
\end{appendices}


