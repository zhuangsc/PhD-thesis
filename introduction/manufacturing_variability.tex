Manufacturing variability or process variation refers to the
power and frequency heterogeneity observed across chips implementing the exact same architecture as a consequence of uncontrollable material differences in the manufacturing process~\cite{Rountree:2012:BDF:2357488.2357648}.
In order to provide homogeneous performance,
chips of the same architecture must hide frequency variability, which can only be achieved via variations in their power consumption.
% However, power variability can still be observed as high as 10\% \cite{Rountree:2012:BDF:2357488.2357648}.
However, in a power constrained environment where all chips need to operate under a certain power cap, this frequency variability can no longer be hidden~\cite{Rountree:2012:BDF:2357488.2357648}, leading to heterogeneous performance.
As a result, a theoretically homogeneous system turns into a heterogeneous one with performance variations of up to 64\%~\cite{Inadomi:2015:AMI:2807591.2807638}.  
Further, this
uneven distribution of delivered performance is specific to each single hardware component, since two nominally identical processors can suffer from different degrees of manufacturing issues.
From the HPC applications perspective, this can cause load imbalances, even if the workload is perfectly balanced, resulting in significantly degraded performance.
To make this problem even worse, since such degradations are hardware specific, it is not possible to design static or hardware agnostic techniques to mitigate this induced new type of load imbalance. 
While ignoring this manufacturing variability leads to performance 
and energy inefficiencies, 
there are opportunities for achieving improvements at the power budgeting or parallel runtime system levels when variability is properly managed~\cite{Teodorescu:2008:VAS:1381306.1382152,Inadomi:2015:AMI:2807591.2807638,Gholkar:2016:PTH:2967938.2967961,Totoni:tech:2014}.
%Currently, state-of-the-art power-aware job scheduling approaches do not consider manufacturing variability %or process variation 
%to manage jobs dispatched across a parallel system.
\par
Manufacturing variability, however, causes  
processors and DRAM memories to react inhomogeneously to power constraints enforced by the system. While already present in current systems, such variability has so far been hidden by varying power consumption to achieve homogeneous performance. 
In fact, existing studies show 
a variation of up to 10\% in power consumption
to deliver the same performance~\cite{Rountree2012}. With the ability to vary power removed by imposing a particular power limit, this variation becomes visible in realized performance~\cite{Inadomi2015}.
Further, this
uneven distribution of delivered performance is specific to each single hardware component, since two nominally identical processors can suffer from different degrees of manufacturing issues.
From the HPC applications perspective, this can cause load imbalances, even if the workload is perfectly balanced, resulting in significantly degraded performance.
To make this problem even worse, since such degradations are hardware specific, it is not possible to design static or hardware agnostic techniques to mitigate this induced new type of load imbalance. 


