\chapter{Introduction}
\label{chap:introduction}
Modern HPC systems make extensive use of their massive amount of CPU core count 
and their peripheral accelerators (GPU, FPGA, ASIC etc.) to achieve a high 
performance~\cite{gropp10, sierra, titan, mn4}.  In order to effectively utilize 
such systems, algorithm designers need to parallelize their problems either by 
hand or relying on compiler or runtime system support. The problems need to be 
meticulously split into smaller chunks that can be executed on the individual 
computational units simultaneously. 

Not all parallelization problems are created equal. For some, denoted as 
\textit{embarrassingly parallel problems}, the task is relatively simple because 
they can be easily solved in a divide and conquer fashion. Each component is 
inherently independent in that it does not require the computational results 
from its counterparts.  

While on the other hand, others oftentimes have non-parallelizable sections that 
create interleaving parallel-sequential-parallel patterns during the execution 
where synchronization is required. It is also a commonplace that parallel 
sections possess dependencies in which case communication inevitably occur.
Such problems are ubiquitous in numerical linear algebra and other fields of 
computational mathematics like matrix multiplication, matrix decomposition, 
eigenvalue solvers, mathematical optimization problems just to name a 
few~\cite{communication0, communication1, communication2, communication3}.

As peripherals, the various types of accelerators are connected through external 
buses like PCIe, NVlink~\cite{nvlink} etc. Necessary data has to be transferred 
from the host CPUs to the accelerators before carrying out any meaningful 
computation. It is prominent among iterative-based numerical methods and with 
the rise of deep neural networks~\cite{ca0, ca1, ca2, ca3}.

Imbalanced synchronization and communication may force the involved computational 
units to waste its computing resources. The scale of the parallel 
system is the primary impact factor of the efficiency of the communication for 
the following reasons.
\begin{itemize}
    \item The physical proximity of the communicating nodes determines the 
        quality of the communication. In a distributed system with nodes 
        scattered at different physical locations, the communication imbalance 
        could create serious bottlenecks.
    \item The need to send data back and forth is alleviated on a shared-memory 
        system where all the computational units have access to entire memory 
        region.  Nevertheless, such systems are inherently limited by size.  
        Another type of underlying memory hierarchy is distributed-memory 
        systems where each node is in possess of a portion of the entire memory.  
        The acquisition of contents from other memory regions has to be resolved 
        by passing messages which could raise contention on the bus system.
\end{itemize}
 
\section{Thesis Objectives and Contributions}
This thesis strives to alleviate the communication by reducing either the 
occurrences of communication points or the quantity of data in the domain of 
iterative numerical methods and deep neural networks, while in the meantime 
retaining the quality of the results the algorithms produce. 

\subsection{Communication Reduction in Conjugate Gradient Method}
The conjugate gradient method solves a linear system in an iterative manner.  
Conventionally, synchronization is needed at the end of each iteration in a 
parallel implementation for some bookkeeping tasks such as checking the 
convergence and applying the residual replacement strategy.  We propose the 
\emph{Iteration-Fusing Conjugate Gradient} which fuses some of the iterations by 
removing the inter-iteration synchronization points within those fused 
iterations and moving the bookkeeping tasks to the end of the last iteration 
from the fusion.  Also we use a task-based parallel programming model to split 
numerical kernels into subkernels to relax data-dependencies. By carrying out 
these two optimizations, our approach allows computations belonging to different 
iterations to overlap if there are no specific data or control dependencies 
between them.
The main contributions of this approach are:
\begin{itemize}
       \item The Iteration-Fusing Conjugate Gradient (IFCG) approach, which aims 
           at aggressively overlapping different iterations. IFCG is implemented 
           by means of two algorithms: IFCG1 and IFCG2.
       \item A task-based implementation of the IFCG1 and IFCG2 algorithms that 
           automatically overlaps computations from different iterations without 
           the need for explicit programmer specification on what computations 
           should be overlapped.
       \item A comprehensive evaluation comparing IFCG1 and IFCG2 with the most 
           relevant state-of-the-art formulations of the CG algorithm.  IFCG1 
           and IFCG2 provide parallel performance improvements up to 42.9\% and 
           41.5\% respectively and average improvements of 11.8\% and 7.1\% with 
           respect to the state-of-the-art techniques and show similar numerical 
           stability.
        \item A demonstration that under realistic system noise regimes IFCG 
            algorithms behave much better than previous approaches. IFCG 
            algorithms achieve an average 18.0\% improvement over the best 
            state-of-the-art techniques under realistic system noise regimes.
\end{itemize}

\subsection{Communication Reduction in Training Deep Neural Network Models}
We describe and evaluate a method to accelerate the training of DNNs by reducing 
the cost of data transfers across heterogeneous high-end architectures 
integrating multiple GPUs. By relying on DNNs tolerance to data representation 
formats smaller than the commonly used 32-bit Floating Point (FP) 
standard~\cite{gupta15, flexpoint17}, we describe how to dynamically adapt the 
size of data sent to GPU devices without hampering the quality of the training 
process.  Our solution is designed to efficiently use the incoming bandwidth of 
the GPU accelerators.
It relies on an adaptive scheme that dynamically adapts the data representation 
format required by each DNN layer and compresses network parameters before 
sending them over the parallel system.
This scheme enables DNNs training to progress in a similar rate as if the 32-bit 
FP format was used.
This work makes the following contributions:
\begin{itemize}
    \item It proposes the {\it Adaptive Weight Precision (AWP)} algorithm, which 
        dynamically adapts the numerical representation of DNN weights during 
        training.  AWP relies on DNNs' tolerance for reduced data representation 
        formats.  It defines the appropriated data representation format per 
        each network layer during  training without hurting network accuracy.

    \item It proposes a new {\it Approximate Data Transfer (ADT)} procedure to 
        compress DNN's weights according to the decisions made by the AWP 
        algorithm.  ADT relies on both thread- and SIMD-level parallelism  and 
        is compatible with architectures like IBM's POWER or x86. ADT is able to 
        compress large sets of weights with minimal overhead, which enables the 
        large performance benefits of our approach.

    \item It evaluates ADT and AWP on two high-end systems: The first is 
        composed of two x86 Haswell multicore devices plus four Tesla GK210 GPU 
        accelerators and the second system integrates two POWER9 chips and four 
        NVIDIA Volta V100 GPUs.  Our evaluation considers the 
        Alexnet~\cite{alexnet}, the VGG~\cite{vgg} and the Resnet~\cite{resnet} 
        network models applied to the ImageNet ILSVRC-2012 
        dataset~\cite{imagenet}.
        Our experiments report average performance benefits of 6.18\% and 
        11.91\% on the x86 and the POWER systems, respectively.
        Our solution does not reduce the quality of the training process since 
        networks final accuracy is the same as if they had been trained with the 
        32-bit Floating Point format.
\end{itemize}

\subsection{Communication Reduction in Deep Learning Model Parallelism}


\section{Thesis Structure}
