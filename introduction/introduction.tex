\chapter{Introduction}
\label{chap:introduction}
Modern parallel systems make extensive use of their massive amount of CPU core 
counts or their peripheral accelerators (GPU, FPGA, ASIC etc.). In order to 
effectively parallelize the problem at hand, algorithm designers need to 
meticulously split the problem into smaller chunks that can be executed on the 
individual computational units. \textbf{\textcolor{red}{expansion needed}}

Not all problems are created equal. For some, denoted as \textit{embarrassingly parallel}, 
the task is relatively simple because they can be easily solved in a divide and 
conquer fashion. Each component is inherently independent in that it does not 
require the computational results from its counterparts. 
\textbf{\textcolor{red}{examples, refs}}

Others on the other hand, oftentimes have non-parallelizable sections that 
create interleaving parallel-sequential-parallel patterns during the execution where 
synchronization is required. It is also a commonplace that parallel sections 
possess dependencies in which case communication inevitably occur.
\textbf{\textcolor{red}{examples, refs}}

As peripherals, the various types of accelerators are connected through external 
buses like PCI, NVlink etc. Necessary data has to be transferred from the host 
CPUs to the accelerators before carrying out any meaningful computation. It is 
prominent among iterative-based numerical methods and with the rise of deep 
neural networks.

Synchronization and communication put the involved computational units on hold 
thus impede the execution progress. The scale of the parallel system is the 
primary impact factor of the efficiency of the communication for the following 
reasons.
\begin{itemize}
    \item The physical proximity of the communicating nodes determines the 
        quality of the communication. In a distributed system with nodes scattered 
        at different physical locations, the communication inbalance could create 
        serious bottlenecks.
    \item The need to send data back and forth is alleviated on a shared-memory 
        system where all the computational units have access to entire memory region. 
        Nevertheless, such systems are inherently limited by size. Another type 
        of underlying memory hierarchy is distributed-memory systems where each 
        node is in possess of a portion of the entire memory. The acquisition 
        of contents from other memory regions has to be resolved by passing 
        messages which could raise contention on the bus system.
\end{itemize}
 
\section{Thesis Objectives and Contributions}
This thesis strives to alleviate the communication by reducing either the 
occurrences of communication points or the quantity of data in the domain of 
iterative numerical methods and deep neural networks, while in the meantime 
retaining the quality of the results the algorithms produce. 

\subsection{Communication Reduction in Conjugate Gradient Method}
The conjugate gradient method solves a linear system in an iterative manner. 
Conventionally, synchronization is needed at the end of each iteration in a 
parallel implementation for some bookkeeping tasks such as checking the 
convergence and applying the residual replacement strategy. 
We propose the \emph{Iteration-Fusing Conjugate Gradient} which fuses some of the 
iterations by removing the inter-iteration synchronization points within those fused 
iterations and moving the bookkeeping tasks to the end of the last iteration from the fusion. 
Also we use a task-based parallel programming model to split numerical kernels 
into subkernels to relax data-dependencies. By carrying out these two optimizations, 
our approach allows computations belonging to different iterations to overlap if 
there are no specific data or control dependencies between them.
The main contributions of this approach are:
\begin{itemize}
       \item The Iteration-Fusing Conjugate Gradient (IFCG) approach, which aims 
           at aggressively overlapping different iterations. IFCG is implemented 
           by means of two algorithms: IFCG1 and IFCG2.
       \item A task-based implementation of the IFCG1 and IFCG2 algorithms that 
           automatically overlaps computations from different iterations without 
           the need for explicit programmer specification on what computations should be overlapped.
       \item A comprehensive evaluation comparing IFCG1 and IFCG2 with the most 
           relevant state-of-the-art formulations of the CG algorithm. 
           IFCG1 and IFCG2 provide parallel performance improvements up to 42.9\% 
           and 41.5\% respectively and average improvements of 11.8\% and 7.1\% with 
           respect to the state-of-the-art techniques and show similar numerical stability.
        \item A demonstration that under realistic system noise regimes IFCG 
            algorithms behave much better than previous approaches. IFCG algorithms 
            achieve an average 18.0\% improvement over the best state-of-the-art 
            techniques under realistic system noise regimes.
\end{itemize}

\subsection{Communication Reduction in Training Deep Neural Network Models}

\subsection{Communication Reduction in Deep Learning Model Parallelism}


\section{Thesis Structure}
%
The remainder of this thesis is organized as follows:

In Chapter \ref{chap:background} we summarize the most widely used designs in modern parallel
architectures and runtime systems.  We also present the most prominent programming models designed 
for such systems and focus on the state-of-the-art in the widely used task-based programming model 
for shared memory architectures.  Furthermore, we discuss the shortcommings in the current state of 
benchmarking in HPC systems.  In the end of the Chapter we present an emerging issue in modern HPC 
system design, the manufacturing variability and how it effects a system's power efficiency.
In Chapter \ref{chap:methodology} we show our experimental platforms, the hardware-software stack
and methodology used to obtain the results presented in this thesis. 
In Chapter \ref{chap:task_based_benchmarks} we present our task-based implementation of the PARSEC 
benchmarks suite.  We show how tasks and dataflow annotations can be used to improve the performance 
and programmability of parallel applications, when compared to other commonly used programming 
paradigms.  Moreover, we discuss why the PARSEC benchmark suite and our implementation offers a better and more realistic testbed for HPC systems. 
In Chapter \ref{chap:power_aware_runtime} we further discuss how manufacturing variability can harm performance and the energy efficiency of an HPC system.  In this Chapter we focus on a runtime solution to mitigate the effects of manufacturing variability.
In Chapter \ref{chap:power_aware_job_scheduling} we move from the runtime to a more system-wide approach to dealing with manufacturing variability.  We show how variability-aware power prediction models can be employed to guide job scheduling decisions, in order to improve the job throughput and power efficient of an HPC cluster.
In Chapter \ref{chap:conclusions} we offer our closing remarks and summarize the contributions and conclusions of this work.  We also discuss the future direction of this work.


