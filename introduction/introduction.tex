
\chapter{Introduction}
\label{chap:introduction}
 
According to Moore's Law the number of transistors in integrated circuits doubles every
two years \cite{Moore:2000:CMC:333067.333074}.  This observation can have different
interpretations, since more transistors mean additional or more complex components.
Computer architects were able to exploit this by designing performance enhancing
components, such as instruction pipelining, branch predictors and memory caches.
Out-of-order execution of instructions, known as Instruction Level Parallelism (ILP), is
also possible by implementing it at hardware level, which allowed computers to execute
multiple instructions per cycle.  All these, along with a constantly increasing clock
frequency, have steadily led to an increase in performance with every new generation of
processors, again roughly every two years.  Moore's observation held true for more than
three decades, however it reached a halting point in the early 2000's.  
 
The post-Moore's Law era brings further implications to computer system design.  Dennard
scaling \cite{1050511}, which relates to Moore's Law, states that as the size of
transistors reduces, their power density stays the same.  This means that the performance
per watt rate had also been increasing exponentially.  As with Moore's law however, this
observation is no longer valid.  Intel's fastest single core processor (Intel Pentium 4),
running at 3.8GHz, reached 100W in power consumption.  The power and thermal dissipation
at higher frequencies make faster processors unfeasible with today's technology.   As a
result, we can no longer rely on transistor technology or increasing clock frequency to
deliver higher performance, as we have reached a point of diminishing returns.
Furthermore, the increasing power demands of processors and the memory subsystem are not
only a significant strain on the budget which often exceeds the cost of purchasing the
machine \cite{Rimal:2009:TSC:1683301.1684085}, but have also reached a point where it is
now arguably the defining limit for further performance enhancements. This is known as the
Power Wall \cite{917539}, which references the obstacle that power and heat pose in modern
processor design.

At the beginning of the 21st century, computer architects took a different direction,
which improved performance significantly and at a feasible power consumption.  The idea
was to use multiple simple processors on the same chip, and exploit parallelism that
algorithms may have, since ILP is inherently limited \cite{Wall:1991:LIP:106972.106991}.
This led to the design of chip multi-processors (CMPs), which can take advantage of
parallelism at application level.  Moreover, the simpler processors that compose a
multi-core chip require less power to operate, while their net benefit in performance is
significantly higher than that of a single more powerful processor.  The rise of the
multi-core era raises new challenges, such as cores contesting for shared resources and
departing from the deterministic nature of sequential algorithmic execution.  

Moreover, the complexity of producing integrated circuits with transistors at such a
microscopic scale has introduced artifacts in the manufacturing process, also known as
manufacturing variability.  These artifacts can effect the transistor switching speed and
leakage, thus their performance and power consumption.   Processor vendors today are
unable to guarantee that their products, even processors of the same chip design, stepping
and firmware, will consume the same amount of power or perform at the same frequency
\cite{Rountree:2012:BDF:2357488.2357648}.  This form of variability is not only observed
among chips, but also between cores within the same die.  Early studies show that
manufacturing variability can cause variation in power consumption up to 10\%
\cite{Rountree:2012:BDF:2357488.2357648} and in performance up to 20\%
\cite{Marathe:2017:ESP:3149412.3149421}, for processors running identical workloads.
Manufacturing variation is also observed to increase with every new generation of
processors \cite{Marathe:2017:ESP:3149412.3149421}, making it an important issue for
current and future processor design.  As a result, modern CPUs are inherently
heterogeneous in terms of either performance or power consumption.  This form of
heterogeneity further complicates programming CMPs, where parallel workloads cannot be
statically distributed.

Even though programming and designing parallel systems requires significant effort, the
raw performance offered by such platforms outweighs the problems.  The excessive
parallelism which is present in a large number of workloads from various domains of
computing, offers the opportunity for performance to scale well beyond the limited number
of cores found on single processor.  Supercomputers, which are used in a variety of fields
- from scientific simulations of astronomical models and genome analysis, to industry for
aerodynamic modeling etc., are composed of hundreds or even thousands of such commodity
processors.  An example of a powerful contemporary system is the MareNostrum 4
supercomputer located in Barcelona, which is ranked as the 22nd fastest in the world, and
3rd in Europe \cite{TOP500}.  It has 153,216 cores and can reach a peak performance of
10.296TFlops/s at 1,632kW of power consumption.  The current fastest supercomputer,
according to the TOP500 list \cite{TOP500}, is Summit, located at Oak Ridge National
Laboratory.  It consists of 2,282,544 cores that can reach a peak performance of
187,659.3TFlops/s at 8,806kW power consumption.  However, the problems observed in a
single chip, also scale exponentially on such enormous systems.  Power, in specific, is a
major limiting factor from moving to the exascale era, since the cost for cooling and
powering a machine consisting of millions of processors, an order of magnitude greater
than contemporary supercomputers, is simply forbidding.  Manufacturing variability can
also become a more serious concern, since it introduces heterogeneous performance or power
consumption among the different nodes.

To deal with increased complexity of designing parallel programs, many new programming
models and paradigms emerged.  They all aim at abstracting architectural details, such as
the memory subsystem, and expressing parallelism at application level.  When programming a
single CMP, the most common and successful approach has been Fork-join models, like Thread
level parallelism (TLP) and the more sophisticated OpenMP \cite{openmp13}.  Parallel work
is divided by the user into multiple entities and run concurrently as different threads of
execution.  This paradigm is common in shared memory systems, where all threads can access
the same memory.  The user however is responsible for synchronizing the memory accesses
and avoid racing conditions.  Writing and reading memory in the wrong order can produce
erroneous results, while poor synchronization strategies can cause bottlenecks and race
conditions.  In distributed memory systems (e.g. HPC clusters), where an application runs
on many different nodes which do not have access to the same memory, programming paradigms
like message passing (MPI \cite{Nagle:2005:MCR:1239662.1239666}) are used.  Programmers
need to explicitly move data from one node to another in order for all processes to have
an consistent view of the data, at all stages of an application's execution.  Data
transfers are costly, in terms of cycles, and can again cause contention and race
conditions, if not used properly.  In both shared and distributed memory systems,
significant effort is required from the user in order to guarantee the correctness of the
application's execution, as well as achieve good performance.  Often, this task requires a
very deep understanding of the programming model and the underlying hardware.  More
sophisticated programming models try to take the burden of the user by providing more
intuitive ways to express parallelism.  Task parallel models for example, allow the user
to abstract parallel work into task constructs.  This task is usually as simple as
annotating certain function or code blocks as tasks, which can then be executed
concurrently
\cite{Fatahalian:2006:SPM:1188455.1188543,Blumofe1995,Bellens:SC2006,Ayguade:2009:DOT:1512157.1512430,Tzenakis:2012:BBD:2370036.2145864,Jenista:2011:OSO:2038037.1941563,Planas:2009:HTP:1572226.1572233,Duran:PPL2011}
Task models often employ dataflow annotations, which allow the user to express memory
accesses and data dependencies between tasks in the form of task arguments
\cite{Jenista:2011:OSO:2038037.1941563,Ayguade:2009:DOT:1512157.1512430,Tzenakis:2012:BBD:2370036.2145864,Duran:PPL2011}.
Models such as these, reduce the need for explicit synchronization.  The execution of
parallel work and memory access synchronization is realized by dedicated runtimes, which
are part of the programming models.  The underlying runtime can also take care of load
balancing parallel work among cores and/or different nodes, remaining transparent to the
user.  In practice, shared and distributed memory models are often mixed, where they deal
with inter- and intra-node parallelization. 


Exploiting the immense power offered by HPC clusters, requires the use of the
aforementioned programming models, in order to express as much parallelism as possible in
one's program.  However, machines such as these have multiple user and require dedicated
software that manage shared resources (such as cores and memory) and the applications that
are queued for execution (typically referred to as jobs).  The most commonly used workload
manager is SLURM \cite{slurm_02}. The resulting ecosystem, which allows us to efficiently
use these high-end machines, consists of a mix of sophisticated hardware and software.
Multicore processors coupled with high bandwidth interconnection, programming models that
allows us to exploit parallelism, runtime systems that take care of parallel execution at
application level, and finally a workload manager, which manages jobs, while efficiently
allocating the system's resources.  In their simplest form, HPC clusters are perceived as
homogeneous machines, where each node runs at the same frequency and power requirements.
Often more complex setups are used in practice, where for example general purpose CPU work
along with GPUs.  Even in this scenario though, it is easy to classify nodes in
homogeneous clusters.  Resource managers and runtime systems are also often used to hide a
system's heterogeneity, by load balancing work dynamically.  Modern multi-core and
multi-node systems, which can be composed of thousands CPUs, can no longer be perceived as
homogeneous systems, neither in terms of CPU frequency nor power consumption, due to
manufacturing variability.  Vendors do not offer any classification of their processors'
variability due to manufacturing issues, making even more difficult for users to identify
and take into account this new form of heterogeneity.  However its impact can be
significant in both power consumption and performance.  Runtime systems and system-wide
workload management software can act as platforms for developing methodologies that both
identify and classify manufacturing variability and mitigating its effects.  


%\subsection{Power Wall}
%Power is becoming a major financial and environmental concern, restricting the compute capacity of High Performance Computing (HPC) systems. 
Today's most power efficient supercomputer operates at 14.1GFLOPS/W~\cite{Green500:2017}, but even if we had a system able to operate at the 50GFLOPS/W rate, which is the limit that some funding agencies have set up for building an exascale machine, the full system would consume several tens of MWatts of power, which constitutes a large economic burden. 
Consequently, a report from the US Department of Energy (DOE)~\cite{ASCAC:tech:2014} identifies energy efficiency as one of the top ten research challenges 
on the road to exascale. % is building energy efficient systems.  
%The current goal of the US DOE is to reach exascale capacity at a 20MW constraint \cite{ashby2010}.  
For similar reasons, the European Union has set up an HPC program % is also concerned and 
focused on building low-power systems based on mobile technology~\cite{rajovic2013}.
%This issue is well known and is often is tackled by sacrificing performance in favor of limiting power consumption.  As a result, it is not uncommon for HPC clusters to operate
%under power constrains.  
\par
An emerging design practice for HPC systems, known as \textit{overprovisioning}~\cite{patki:2013:eho:2464996.2465009}, 
is to have more nodes than the maximum power budget could feed if run at peak capacity, in contrast to traditional approaches, which are focused in having enough
power even when all nodes run at their peak. 
Overprovisioning is driven by the observation that  most applications in practice never reach peak power and hence do not fully utilize the available power envelope.
In such \textit{overprovisioned} systems, we can lower the average power provisioned to each node, allowing us to power more nodes within the same power budget. 
%Additionally, we can redistribute remaining unused power resources to existing nodes by increasing their power allocation for periods of time when additional power is helpful in speeding up computation, e.g., when computations are on the critical path. Combined, both options allow us to shorten execution time and/or improve turnaround time.
%
%
%either the extra nodes can take advantage of this available power, or it can be Alternatively, job turnaround time can be increased by redistributing power among all the available nodes.
%This approach is made possible by recent developments in hardware design that enable power management and power capping from user space, as this is necessary to efficiently manage power 
%as a limited and shared resource.
\par
The resulting need for reducing hardware power consumption has started to force computer architects and vendors to include power capping capabilities in their hardware designs. This allows applications to more efficiently exploit their entire power envelope of a system, while  guarding the system against intermediate power spikes. Prior work has shown that this can lead to significant performance benefits~\cite{patki:2013:eho:2464996.2465009,conductor2015}.



%\subsection{Manugacturing Variability}
%Manufacturing variability or process variation refers to the
power and frequency heterogeneity observed across chips implementing the exact same architecture as a consequence of uncontrollable material differences in the manufacturing process~\cite{Rountree:2012:BDF:2357488.2357648}.
In order to provide homogeneous performance,
chips of the same architecture must hide frequency variability, which can only be achieved via variations in their power consumption.
% However, power variability can still be observed as high as 10\% \cite{Rountree:2012:BDF:2357488.2357648}.
However, in a power constrained environment where all chips need to operate under a certain power cap, this frequency variability can no longer be hidden~\cite{Rountree:2012:BDF:2357488.2357648}, leading to heterogeneous performance.
As a result, a theoretically homogeneous system turns into a heterogeneous one with performance variations of up to 64\%~\cite{Inadomi:2015:AMI:2807591.2807638}.  
Further, this
uneven distribution of delivered performance is specific to each single hardware component, since two nominally identical processors can suffer from different degrees of manufacturing issues.
From the HPC applications perspective, this can cause load imbalances, even if the workload is perfectly balanced, resulting in significantly degraded performance.
To make this problem even worse, since such degradations are hardware specific, it is not possible to design static or hardware agnostic techniques to mitigate this induced new type of load imbalance. 
While ignoring this manufacturing variability leads to performance 
and energy inefficiencies, 
there are opportunities for achieving improvements at the power budgeting or parallel runtime system levels when variability is properly managed~\cite{Teodorescu:2008:VAS:1381306.1382152,Inadomi:2015:AMI:2807591.2807638,Gholkar:2016:PTH:2967938.2967961,Totoni:tech:2014}.
%Currently, state-of-the-art power-aware job scheduling approaches do not consider manufacturing variability %or process variation 
%to manage jobs dispatched across a parallel system.
\par
Manufacturing variability, however, causes  
processors and DRAM memories to react inhomogeneously to power constraints enforced by the system. While already present in current systems, such variability has so far been hidden by varying power consumption to achieve homogeneous performance. 
In fact, existing studies show 
a variation of up to 10\% in power consumption
to deliver the same performance~\cite{Rountree2012}. With the ability to vary power removed by imposing a particular power limit, this variation becomes visible in realized performance~\cite{Inadomi2015}.
Further, this
uneven distribution of delivered performance is specific to each single hardware component, since two nominally identical processors can suffer from different degrees of manufacturing issues.
From the HPC applications perspective, this can cause load imbalances, even if the workload is perfectly balanced, resulting in significantly degraded performance.
To make this problem even worse, since such degradations are hardware specific, it is not possible to design static or hardware agnostic techniques to mitigate this induced new type of load imbalance. 




\section{Thesis Objectives and Contributions}

In this thesis we study manufacturing variability on modern processors and propose power-
and variability-aware scheduling policies at runtime and system wide Job Scheduling
levels.  Our goal is to mitigate the effects of manufacturing variability in modern
processors in order to optimize their performance and power efficiency.  Our approach,
treats power as a shared resource which needs to be carefully managed and allocated to
parallel workloads.  We propose two different approaches, one at application runtime level
and one at system wide resource management level.  The runtime approach deals with power
constraint systems, where it monitors the application's performance during execution and
redistributes power and work between the sockets of a node, till an optimized setup is
reached.  In order to evaluate our runtime approach, we needed a benchmark suite
representative of modern workloads and implemented using our runtime system.  For this we
implement our own task-based version of the PARSEC benchmark suite, using OpenMP 4.0
tasks.  The system wide approach tries to maximize the throughput of the system while
minimizing power consumption.  Part of our objective is also to understand how our methods
affect applications that are actually being used in modern HPC systems.


\subsection{Benchmarking with Realistic Workloads}

Benchmarking is a vital part of evaluating experimental software and hardware.  In HPC,
benchmarking is usually restricted to small kernel applications.  The reasoning behind
this trend however is that larger applications consist of smaller kernel ones.  Although
this is true, it is not always the case that performance can only be gained by
parallelizing or optimizing these kernel applications.  Coarser grain parallelism also
plays an important factor in improving the performance of applications used in today's
computing \PARSEC{} \cite{Bienia:PhD2011}.    In this thesis, we aim to evaluate the
benefits of task-based parallelism beyond the scope of HPC kernels, focusing on a set of
parallel applications representative of a wide range of domains from HPC to desktop and
server applications.  To do so, we apply task-parallelism strategies to the \PARSEC{}
benchmark suite~\cite{Bienia:PhD2011} and compare them in terms of programmability and
performance with respect to the fork-join versions contained in the suite. 
The main contributions of are:
\begin{itemize}
	\item We apply task-based parallelization strategies to 10 \PARSEC{} applications. 
	\item We fully evaluate them in terms of performance, considering different scenarios 
(from 1 to 16 cores) and achieving average improvements of 13\%. 
In some particular cases, the improvements reach 42\%. 
	\item We provide detailed programmability metrics based in lines of code, achieving an 
average reduction of 28\% and reaching a maximum of 81\%.
\end{itemize}

Our superscalar version of the \PARSEC{} benchmark suite (PARSECSs), fills the gap in the
evaluation methodology usually used in HPC, offering a set of applications actually used
in todays computing. The implementation offered also uses the state-of-the-art concepts
and models in parallel programming.

\subsection{Variability-Aware Load-balancing at Runtime Level}

To deal with manufacturing variability, we present a runtime guided hardware/software
reconfiguration approach that effectively mitigates the effects of inhomogeneous hardware
behavior in low power environments.  Further, we demonstrate that classical work stealing
and load balancing techniques~\cite{Blumofe1999, Blumofe1995, Ravichandran2011, Zheng2011}
are insufficient to mitigate this performance issue.  In the context of a NUMA node
composed of several multi-core sockets, our technique is able to efficiently distribute a
total node power budget among the node's different sockets, while also adjusting their
corresponding concurrency levels.  In order to enable this, our approach dynamically
selects the best power/concurrency level for each socket involved in the computation by
performing a light weight initial training phase.  This initial training phase selects the
optimal power/concurrency level to be assigned to each socket to reduce applications' load
imbalance induced by power/performance inhomogeneity and thus increase performance. 

The contributions are as follows:
\begin{itemize}
	\item We provide a precise description of the limitations of current, state of the art 
load balancing techniques when dealing with inhomogeneous hardware behavior under node-level 
power limits. 
  \item We demonstrate how uneven power and thread assignments to sockets can mitigate the 
inhomogeneous hardware behavior in dual socket NUMA nodes, resulting in up to 30\% increased 
performance for some applications.
	\item We describe a dynamic runtime technique to discover the optimal power/concurrency 
assignment for each application on a given parallel machine that provides up to 22\% 
performance improvements for some applications. 
\end{itemize}

%Chapter \ref{chap:power_aware_runtime} describes the above contributions in more detailed and offers an evaluation of the runtime approach.

\subsection{Power Variability Prediction-driven Job Scheduling}

We propose two variability-aware job scheduling policies to deal with manufacturing
variability at system-wide level, by introducing power- and variability-awareness to the
cluster resource and job manager.  Workloads at the HPC system level are managed by job
schedulers that allocate resources to dispatched jobs.  Such jobs can run on distributed
memory scenarios and, in this context, MPI~\cite{Nagle:2005:MCR:1239662.1239666} is the
most common approach to handle distributed memory communications.  It is usually coupled
with a shared memory programming model, like OpenMP~\cite{openmp13} or
similar~\cite{10.1007/978-3-540-85261-2_5}. 

Either across nodes or within a shared-memory node, both job and runtime schedulers deal
with the resource allocation problem, albeit at different levels, offering opportunities
to manage power consumption.  Indeed, examples of power-aware systems  that offer
solutions either at the job scheduling
\cite{Gholkar:2016:PTH:2967938.2967961,7515666,Ellsworth:2015:DPS:2807591.2807643,Etinski2012615}
or at runtime system  \cite{Gholkar:2016:PTH:2967938.2967961,
Chasapis:2016:RMM:2925426.2926279,Totoni:tech:2014,Teodorescu:2008:VAS:1381306.1382152,Inadomi:2015:AMI:2807591.2807638}
levels already exist in the literature.

This work goes beyond the state-of-the-art by proposing job scheduling policies driven by
variability-aware power prediction models.  We extend power-aware scheduling and power
prediction models to deal with manufacturing variability,  producing two novel
variability- and power-aware job scheduling policies.  We consider the power consumption
of the CPU, since it accounts for more than 50\%
\cite{ShuaiwenSong:2009:EPA:1572226.1572228} of the total node's power consumption.  Our
Policies rely on two different models and leverage their power requirement predictions of
individual parallel jobs to make scheduling decisions that maximize performance while
reducing energy consumption.  Many different variability-agnostic power and energy
prediction models have been proposed
~\cite{Bircher:2012:CSP:2196827.2196987,Bertran:2012:SEC:2457472.2457499,Bertran:2010:DRP:1810085.1810108,Goel:2010:PSP:1909624.1909734,Isci:2003:RPM:956417.956567}
and are often employed to manage power distribution on clusters or mitigate the effects of
manufacturing variability
~\cite{Chasapis:2016:RMM:2925426.2926279,Inadomi:2015:AMI:2807591.2807638,Gholkar:2016:PTH:2967938.2967961,Ellsworth:2015:DPS:2807591.2807643,Bailey:2015:FLP:2807591.2807637,Teodorescu:2008:VAS:1381306.1382152,Totoni:tech:2014}.
Our work shows how variability-aware power prediction models can be effectively used to
guide job scheduling policies and bring significant benefits with respect to the
variability-agnostic ones.
\par
In particular, this work makes the following contributions:
\begin{itemize} 

	\item Two new variability-aware power prediction models.  The first model assumes power
variability to impact all applications equally and it is based on executions of a single
benchmark on different sockets to measure the power consumption variability across them.
Given the power profile of the targeted application obtained from a previous run on a
certain socket, the model applies the measured variability ratios to predict its power
consumption on all sockets.  The second model extends the Performance Monitoring
Counters-based \acrlong{pmc} (PMC) approach to take power consumption variability into
account.  PMCs are used to measure the activity of individual architectural components
while the targeted application is running. Using a linear model trained off-line by
running a reduced set of benchmarks on all sockets, the model predicts the power
consumption of each architectural component for a certain application.  

	\item Two power- and variability-aware job scheduling policies that optimize job turnaround 
time and energy efficiency while respecting a system-wide power budget.  prediction model.
Unlike previous work that does not consider variability during job scheduling decisions
~\cite{Inadomi:2015:AMI:2807591.2807638,Teodorescu:2008:VAS:1381306.1382152,Ellsworth:2015:DPS:2807591.2807643,Gholkar:2016:PTH:2967938.2967961},
our policies use variability-aware prediction model to guide scheduling.

	\item A complete evaluation of the two variability-aware policies via a discrete event 
simulator.  We implement additional scheduling policies for our evaluation, which
represent traditional and state-of-the-art practices used in today's HPC systems.  Our
evaluation demonstrates how variability-aware policies achieve energy savings up to
\MaxEnergy\% and job turnaround time reductions up to \MaxJTT\%, considering different
power budgets and two workload traffic scenarios (bursty and heavy).

\end{itemize}    	



\section{Thesis Structure}

The remainder of this thesis is organized as follows:

In Chapter \ref{chap:background} we summarize the most widely used designs in modern parallel
architectures and runtime systems.  We also present the most prominent programming models designed 
for such systems and focus on the state-of-the-art in the widely used task-based programming model 
for shared memory architectures.  Furthermore, we discuss the shortcommings in the current state of 
benchmarking in HPC systems.  In the end of the Chapter we present an emerging issue in modern HPC 
system design, the manufacturing variability and how it effects a system's power efficiency.
In Chapter \ref{chap:methodology} we show our experimental platforms, the hardware-software stack
and methodology used to obtain the results presented in this thesis. 
In Chapter \ref{chap:task_based_benchmarks} we present our task-based implementation of the PARSEC 
benchmarks suite.  We show how tasks and dataflow annotations can be used to improve the performance 
and programmability of parallel applications, when compared to other commonly used programming 
paradigms.  Moreover, we discuss why the PARSEC benchmark suite and our implementation offers a better and more realistic testbed for HPC systems. 
In Chapter \ref{chap:power_aware_runtime} we further discuss how manufacturing variability can harm performance and the energy efficiency of an HPC system.  In this Chapter we focus on a runtime solution to mitigate the effects of manufacturing variability.
In Chapter \ref{chap:power_aware_job_scheduling} we move from the runtime to a more system-wide approach to dealing with manufacturing variability.  We show how variability-aware power prediction models can be employed to guide job scheduling decisions, in order to improve the job throughput and power efficient of an HPC cluster.
In Chapter \ref{chap:conclusions} we offer our closing remarks and summarize the contributions and conclusions of this work.  We also discuss the future direction of this work.



