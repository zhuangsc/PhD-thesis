
Knowing an applications power requirements can be invaluable information when making decisions
on when an where to run it in a large HPC cluster.  Relying on user supplied information is not
enough, since this information can be both difficult to obtain but also unreliable.
An alternative is to train analytical prediction models to get the information. 
Power prediction models have been extensively studied over the years.  In particular, models based on PMC
have been very successful in  predicting power consumption~\cite{Bircher:2012:CSP:2196827.2196987,Bertran:2012:SEC:2457472.2457499,Bertran:2010:DRP:1810085.1810108,Goel:2010:PSP:1909624.1909734,Isci:2003:RPM:956417.956567}.  
However, none of the models found in the literature consider manufacturing variability. 
In this work, we demonstrate that it is possible to make PMC prediction models aware of the 
manufacturing variability and accurately predict the socket and application specific power 
consumption, insteadof making the same prediction for all sockets.  
Our PMC-based model is based on the work of Bertran et al.~\cite{Bertran:2012:SEC:2457472.2457499,Bertran:2010:DRP:1810085.1810108}, which aims at providing insight into the way individual architectural components influence power consumption.
Our PMC-based model extends Bertran's model to account for manufacturing variability in terms of power consumption.
The original model requires carefully crafting micro-benchmarks that isolate activity per architectural component, in order to train it.  We show that comparable results can be obtained 
by training the model with a small set of kernel applications and a microbenchmark that stresses the memory unit.
The benefit of using the suggested set of applications for training the model, is that it is a more
portable solution, compared to Beltran's original set of microbenchmarks.  In Beltran's work, 
the microbenchmarks need to be designed so that only certain architectural components are stessed
by each microbenchmark.  This is achieved by carefully implementing an assembly code and considering
all the architectural details of the underlying machine.  Although our approach will suffer a small
penalty in precision, there is not need of modification when run on a different machine.
%, which makes their approach architectural dependent.
Our approach intends to predict power consumption variability while remaining easy to 
deploy on any system.
The implementation, application and evaluation of our proposed analytical
analytical prediction model is discussed in Chapter \ref{chap:power_aware_job_scheduling}. 

