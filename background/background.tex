
\chapter{Background}
\label{chap:background}

In this Chapter we provide the necessary background context and the state-of-the-art
related to this thesis.  In Section \ref{sec:parallel_systems} we describe the different
architectures found in today's parallel systems.  We divide them into two main categories,
based on the way memory is viewed by the individual computing units.  We then describe how
a computing cluster is designed and explain the distinction between homogeneous and
heterogeneous clusters.  In Section \ref{sec:parallel_programming_models}, we present the
different programming models used to program shared and distributed memory machines, as
well as the challenges users face when writing parallel applications.  We then discuss how
more sophisticated programming models can help users write more efficient and maintainable
parallel code, when coupled with a runtime system.  In Section \ref{sec:hpc_benchmarking},
we discuss the limitations of contemporary benchmarking suites and methodologies, and make
a case for moving to a benchmarking suite that better represents workloads run on today's
HPC systems, implemented in a state-of-the-art parallel programming model.  In Section
\ref{sec:cluster_management}, we describe the software used to manage the workload and the
resources on an HPC cluster.  Manufacturing variability, which causes same model
processors to run at varying frequencies and power consumption is presented and discussed
in Section \ref{sec:process_variability}.  The state-of-the-art in power managing HPC
clusters and improving the power efficiency of parallel runtime systems is presented in
Section \ref{sec:power_management}.

\section{Parallel Systems}
\label{sec:parallel_systems}
With the stagnation of processor frequency and the inherent limitation of ILP in computer
programs, computer architects turned to multi-core processor chip design in order to
exploit parallelism at application level.  Both hardware design and software
implementation for exploiting the parallelism available by the multiple cores brings
significant new challenges.  Traditionally, computer programs were designed in an
sequential algorithmic manner.  However, in a parallel environment, the tasks performed by
a program need to be divided into smaller ones, which can be concurrently executed.  This
is known as thread or task level parallelism (TLP).  In an ideal scenario, a program can
be parallelized in an equal number of concurrent tasks, as the number of available cores.
If a system has $N$ cores and the sequential version of the program run in $T$ seconds,
then the expected speedup would be $N*T$.  This is known as linear scaling.  However, in
practice this is rarely the case.  Even if an algorithm is embarrassing parallel, meaning
that it can easily be divided in $N$ tasks, the speedup may be near-linear because the
application may become bound by memory or I/O.   

Understanding the underlying hardware, is key in order to achieve good performance.  For
example, parallel tasks operate in smaller segments of data than their sequential
counterpart.  This can lead in better utilization of the memory hierarchy and more
efficient use of the data replacement policies.  In such cases, performance can achieve
super-linear speedup, meaning that it is possible to surpass even linear scaling.  The
importance of efficiently utilizing the memory organization on a parallel system is
apparent from the above example.  As such, the most common distinction between parallel
systems is the way memory is organized and viewed by different cores.  There are two main
memory schemes used, shared memory and distributed memory.  

\subsection{Shared and Distributed Memory Systems}
In shared memory systems, all cores can access the entire memory using the same physical
address space.  Typically, modern processors feature multiple cores, each with access to a
private cache and an interface connection to the DRAM subsystem.  A common design for
contemporary processor features private instruction and data L1 caches to each core, an
additional L2 private cache and a shared L3 cache between all cores.  It is also possible
to only have two levels of caches, in which case L2 cache is shared and connects to the
DRAM, instead of the L3.  The shared cache memory (be it L2 or L3) is also referred to as
Last Level Cache (LLC). 

If all cores can access any arbitrary memory location with the same speed (latency and
bandwidth), then such a system is referred to a Uniform Memory Access (UMA).  If however,
the physical location of a core influences the cost of accessing memory, this system is
referred to as Non-Uniform Memory Access (NUMA).  This design is most common in practice
fore modern processors.  A single unit can feature two or more sockets on the motherboard,
mounted with a multi-core processor each.  Each processor has it's own cache hierarchy,
with the LLC connected to the DRAM memory via a cache-coherent network interconnection.
Because memory access time depends whether data is present in the local cache hierarchy of
each processor, such a system is a NUMA one.        

In a distributed memory system, each processor has its own physical memory address space.
All processors can communicate with each other via a network interconnection and can
exchange data through it.  The network bandwidth and latency are of uttermost importance
for the systems performance, since it can act as a bottleneck on large scale systems,
where multiple processors may transfer data through the same interconnection.  As such,
the network topology is very important for such systems.  Connection between processors
can be point-to-point links or use dedicated switching hardware, grouping processors
together.  For the interconnection network, low latency and high throughput protocol is
used, like Infiniband. 

\subsection{HPC Cluster Design}
\label{sec:cluster_design}

HPC cluster typically consist of hundreds or thousands of processors and cores, offering
immense parallelization and computing power.  Summit, the current faster supercomputer,
according to TOP500 list \cite{TOP500}, consists of 2,282,544 cores.  MareNostrum, one of
Europe's largest supercomputers \cite{TOP500}, consists of 19,440 cores.  Multiple
computing units, referred to as nodes from this point on, are connected together via a
network interface, from commodity ones like Ethernet to high-throughput ones like
Infiniband.  In practice, large HPC clusters feature a combination of shared and
distributed memory system design. Each node features multiple UMA or NUMA sockets.  Nodes
use the network interconnection to transfer data between them. 

\subsubsection{Homogeneous vs Heterogeneous Systems}
\label{sec:cluster_homogeneity}
An HPC cluster may only feature same type processors.  Such a system is known as
homogeneous, referring to the uniform computing capacity of all nodes.  General purpose
processors offer a good option for most problem classes, however specialized hardware,
like GPUs, are better candidates for certain problems (e.g. linear algebra and machine
learning) and act as accelerators.  An alternative to the homogeneous system design is
combining different type of computing units.  As an example, a system can feature NUMA
nodes general purpose processors and additional nodes with multiple GPUs.  This design
approach is known as heterogeneous.

Apart from accelerating certain problem cases, heterogeneous system design is also
considered for power efficiency.  Typically, smaller cores are more power efficient than
faster ones.  The GPU approach is to offer a hundred times more cores than processors.
These cores are slower but more power efficient than those of a processor.  Emerging
processor architectures explore the potential of designing multi-core chips where not all
cores have the same computing capacity.  Such processors are referred to as Asymmetric
Multi-core (AMC) processors.   ARM big.LITTLE is an AMC processor design, steered towards
power efficiency.  \emph{Bigger},faster cores are combined with \emph{smaller}, slower but
more power efficient ones.  The reasoning behind this design is to offer fast cores for
running tasks in the critical path of the application, while the smaller ones can offer
more parallelism at a smaller power cost, for non-critical tasks.  Although heterogeneous
systems offer benefits in power efficiency and performance, they are more difficult to
program compared to homogeneous ones.  Since in an heterogeneous system not all processing
units perform the same, evenly distributing work among cores is not a good option.  Faster
cores will finish earlier and remain idle, waiting for the slower ones.  Either the
programmer needs to explicitly distribute work in a fashion that will not create
bottlenecks.  Alternatively, dedicated software can act as a scheduler to dynamically
load-balance work between cores.   

Heterogeneity is not always a deliberate choice.  Due to manufacturing issues, processors
of the same model, stepping and firmware can still demonstrate variability in both
performance and power consumption \cite{Marathe:2017:ESP:3149412.3149421,Rountree2012}.
As such, a system that is expected to act as homogeneous is in fact heterogeneous.
Dynamic load-balancing is again required to mitigate performance issues, while power needs
to also be managed for more efficient use by the available cores.  Software solutions,
like the ones proposed by this thesis, can be employed to deal with this type of
heterogeneity and mitigate its negative effects.

\section{Parallel Programming Models}
\label{sec:parallel_programming_models}

Programming large parallel machines is considered a job for expert users.  To ease the
programming effort and make parallel machines accessible a lot of parallel programming
models have been proposed.  The main goal of any programming model is to offer the means
to express the available parallelism in an application.  Additional factors that
distinguish a programming model is how the underlying machine is abstracted and how the
execution and synchronization of parallel work is managed.  

\subsection{Shared and Distributed Parallel Memory Models}
An important factor that programming models need to handle is the underlying memory
system.  How cores access memory can influence the cost of their communication.  In shared
memory programming models, the most common programming paradigm is fork-join.  Workloads
are decomposed into smaller ones and a new thread of execution is spawned for each.  Since
all cores share the same physical address space, threads can access memory and communicate
a very low cost, but synchronizing accesses is very important to maintain a consistent and
correct view of the memory among all threads.  This threading model is the most prevalent
programming model for shared memory machines, while many other shared memory models derive
from it \cite{openmp13,Blumofe1995,Reinders:2007:ITB:1461409,Kale:1993:CPC:165854.165874}.  

In distributed memory systems, execution threads, usually referred to as processes, have a
private memory.  Workload decomposition needs to take account that data needs to also be
segmented.  Each process only works on the data segment it has a copy of, but after work
completion, results need to be aggregated.  A common approach to achieve communication is
through a message passing library, like MPI \cite{Nagle:2005:MCR:1239662.1239666}.  Such
libraries, offer the communication primitives to transfer send data from one node to
another, in a point-to-point fashion or broadcast to everyone.

Note that the distinction between shared and distributed memory programming models concern
the way memory is abstracted and presented to the user.  For example and MPI application
can use the shared memory subsystem to exchange messages, within the same node or
processor.  Other models, like UCA \cite{El-Ghazawi:2006:UUP:1188455.1188483} can offer a
unified memory view to the user, where the underlying library implementation can take care
of moving data between different address spaces.  In the Futures programming model certain
values are passed around without actually having been evaluated, until a callback
mechanism is called when their value is required by the program.  Futures is a widely used
shared memory programming model, implemented even in the standard C++ and Boost
\cite{Schling:2011:BCL:2049814} libraries.  However, distributed memory implementations
also exist \cite{Reinders:2007:ITB:1461409,DFChasapis}.  In practice, hybrid approaches
are often used.  Shared memory models, like OpenMP, can be coupled with a distributed
memory one, like MPI.  Inter- and intra-node communication and parallelization is handled
by the corresponding model, exploiting the best both have to offer.

\subsection{Synchronization and Parallel Programming Challenges}
\label{sec:par_prog_challenges}

Parallel execution brings new challenges for users.  In sequential execution the order
commands run is well defined by the user.  Contrary, in a parallel environment commands
may run concurrently and different threads of execution race for the shared resources
(e.g. memory) on the system.  Competing for system resources needs to be managed by the
user or dedicated software, such as a runtime system.  If not synchronized properly,
parallel execution threads may starve or in the case of memory, wrong access order may
violate RAW dependencies.  These situations are referred to as /emph{race conditions}.
Different programming models offer varying solutions to the synchronization problem.
Parallel programming models offer solutions like barrier primitives to define these
synchronization points.  In a shared memory environment, where system resources are shared
between threads, synchronization is achieved by using primitives like locks and
semaphores.  These primitives guarantee that a resource will be accessed by only one
thread at a time.  However, which thread and in which order a resource is accessed needs
to be defined by the user.  The order and manner of access is especially important for the
memory system, since racing threads may produce wrong results or evict memory which is in
use by another thread.  In microarchitectural level, where memory instructions require
multiple cycles to complete, atomic instructions guarantee that data will not be accessed
before the instruction completes.  The fork-join model, followed by many shared memory
models (Pthreads,OpenMP), also dictates that threads need to synchronize when joining.  

In distributed memory environments, each process is typically operating on a separate copy
of the data.  Occasions may rise where a process needs to broadcast its results or share
them with a different process.  Synchronization is achieved by transferring the data
through the network, with the use of a library like MPI.  Again, the responsibility for
synchronizing processes is the user's.  In a distributed memory environment, where data is
transferred typically over a network interface, the user needs to also consider the cost
of doing so.  Decomposing the workload into large processes is general preferable, since
communication among processes incurs significant overheads.  Although distributed memory
models can easily abstract and be used with shared memory machines, it is generally
preferable to combine them with shared memory programming models.  The shared memory model
paradigm is more efficient in a shared memory environment, since it does not require
explicitly moving data between cores, causing less overhead.  As already stated, this is a
hybrid approach, where shared memory models are used for inter-node communication and
distributed ones for the intra-node interactions.  

Synchronization is not the only challenge users have to face in parallel environments.
Decomposing the workload into smaller ones, which can be run concurrently, is also
critical.  This step defines the available parallelism in the application, as well as the
data movement and synchronization between different threads and processes.  Moreover, the
homogeneity of the threads/processes workload is also very important.  The user must
consider the distribution of the workload and the architecture of the underlying machine
if he is to fully utilize the available parallelism and the machine's full processing
power.  Statically and evenly distributing inhomogeneous workload will cause certain cores
or nodes to finish before others and remain idle.  Alternative decomposition strategies
can be more efficient, but harder to implement.

\subsection{Asynchronous Tasks and Dataflow Model}
\label{sec:task-model}
Different parallel programming models compete in providing intuitive and novel ways to
express parallelism.  A very successful parallel programming paradigm is task-based
parallelism.  Tasks offer an easy and abstract way to express parallelism.  The OpenMP
4.0~\cite{openmp13}, a widely used programming standard for shared memory machines, allows
the user to annotate functions that can be run asynchronously.  Other programming models,
like Cilk~\cite{Lee:2013:OPP:2486159.2486174}, allow the user to implement parallel
functions, through library calls.  Task-based models require the programmer to synchronize
data accesses between competing parallel tasks, with synchronization primitives.  They are
also typically coupled with a runtime system, implemented as a library, which deals with
task creation, synchronization and load balancing.  More sophisticated task-based models
also offer tools to simplify task synchronization, and allow the runtime to be aware of
the order tasks need to execute, while respecting data dependencies between them.
Explicitly expressing the execution order of tasks enables the runtime to make less
conservative decisions when scheduling tasks, essentially making dynamic load balancing
more efficient.  It also supports dataflow annotations that describe data dependencies
among tasks. This information can be used by the runtime system to synchronize task
execution.  

An intuitive way to express task data dependencies is the dataflow model, which is
implemented by the most widely used programming models, like OpenMP 4.0\cite{openmp13}.
In the dataflow model, the user is able to express the data footprint of a task, typically
in the form of task arguments.  The user also needs to specify whether an argument is
going to be read as input or written to as output, or both as input-output.  Task
arguments are translated into a memory addresses at runtime and the dataflow relations, as
defined by the user, can be used to construct a task dependency graph (TDG), which is an
acyclic directed graph describing the dataflow relations between all available tasks.  The
nodes on such a graph represent the tasks queued for execution and the directed edges
represent the data dependencies between the tasks.   A task is ready to execute when there
are no more input dependencies (input edges).  When a task finished, it's outgoing edges
are removed and the graph is checked again for tasks that may now be free of dependencies.
    
\begin{figure}[ht!]%
	\begin{lstlisting}
void load() {
	int i = 0;
	while( load_image(image[i]) ) {
		#pragma omp task in(image[i]) 
		                 out(seg_images[i])
		seg_images[i] = t_seg(image[i]);
		#pragma omp task in(seg_images[i])
				             out(extract_data[i])
		extract_data[i] = t_extract(seg_images[i]);
		#pragma omp task in(extract_data[i])
		                 out(vectoriz_data[i])
		vectoriz_data[i] = t_vec(extract_data[i]);
		#pragma omp task in(vectoriz_data[i]) 
		                 out(rank_results[i])
		rank_results[i] = t_rank(vectoriz_data[i]);
		#pragma omp task in(rank_data[i]) 
		                 out(outstream)
		t_out(rank_data[i], outstream);
		i++;
	}
	#pragma omp taskwait
}
	\end{lstlisting}
	\caption{\texttt{Ferret} implementation in OpenMP 4.0/\OMPSS{}}%
	\label{lst:ferret-ompss}%
\end{figure} 

Figure \ref{lst:ferret-ompss} shows a simplified version of the \texttt{ferret} benchmark
implemented in \OMPSS{}~\cite{Duran:PPL2011}.  \OMPSS{}~is an extension to the OpenMP 4.0
model with similar syntax and some additional features like socket-aware scheduling for
NUMA architectures.  The \texttt{ferret} application is parallelized with a pipeline
model, where each task is a pipeline stage and the dataflow relations direct the order of
execution of these stages.  The user can use pragma directives to identify functions that
should run asynchronously.  These task pragmas can have dataflow relations expressed with
the use of \texttt{in, out} and \texttt{inout} annotations. These declare whether a
variable is going to be read, written or both by the task.  An underlying runtime system
is responsible for scheduling tasks, track dependencies, balance the load  among available
threads and ensure correct order of execution, as dictated by the dataflow relations.  In
our example data dependencies will force tasks spawned in the same iteration to run in
sequential order, while tasks from different iterations can run concurrently.  An
exception is \texttt{t\_out} which shares a common output between all instances,
\texttt{outstream}, to store the final results of \texttt{ferret}.

Few studies exist that examine the performance of task parallelism compared to other
models.  \cite{10.1007/978-3-540-85261-2_5} evaluate OpenMP tasks by implementing a
few small kernel applications using the new OpenMP task construct.  Their evaluation tests
the model's expressiveness and flexibility as well as performance.  \cite{Podobas503167}
compare three models that implement task parallelism, Wool, Cilk++ and OpenMP.  They
compare their performance using small kernels, as well as some microbenchmarks aimed to
measure task creation and synchronization costs.  They show that Cilk++ and Wool have
similar performance, while they outperform OpenMP tasks for fine grain workloads.  On
coarser grain loads, all models have matching performance with OpenMP gaining in one case,
due to superior task scheduling.

BDDT~\cite{Tzenakis:2012:BBD:2370036.2145864} is a task-based parallel model, very similar
to OmpSs, that also uses a runtime to track data dependencies among tasks.  BDDT uses
block-level argument dependency tracking, where task arguments are processed into blocks
of arbitrary size, which is defined by the user.  This offers some flexibility when
tracking dependencies of arrays, without the need to modify the memory layout, while also
maintaining precision (depending on the chosen block-size).  Moreover, it offers
additional syntax semantics to exclude certain arguments from the dependency analysis,
further reducing the overhead of online dependency tracking by reducing the size of the
dependency graph.  BDDT is shown to outperform loop constructs implemented using OpenMP. 

\subsection{Parallel Runtime Systems}  
In Section \ref{sec:par_prog_challenges} we discuss the challenges users need
to face when programming parallel machines.  Most parallel programming models
today, such as the task-based dataflow model presented in Section
\ref{sec:task-model}, implement runtime systems that deal with some of these
challenges in different ways.  Some runtime systems may only offer
synchronization primitives, like barriers and locks, or implement data transfer
primitives, like broadcasting.  It is often the case though that a runtime
system offers additional functionality to take some of the burden of the user's
shoulders.  A common key feature is managing the parallel workload, distribute
it among cores, instead of depending on the user to statically divide and
distribute it.  Dynamically handling a parallel workload is an easy task for a
runtime, which can redistribute work to idle cores or nodes.  This feature is
often referred to as dynamic load balancing, and allows the user to ignore some
of the underlying architectural details, like the heterogeneity of the cores or
nodes and the manufacturing variability of CPUs.  Apart from relieving the
programming effort, these features make the code more portable and maintainable
across different machines. 

The versatility of how a runtime system can be exploited to improve
performance, energy consumption or expand a model's functionality is
demonstrated by the sheer number of different approaches and techniques
developed by research centers and industry.  Chronaki et al. \cite{7762236}
improve performance of task-based models on asymmetric multi-core architectures
by identifying task that are in the critical path, and scheduling such tasks on
the faster processing units of the machine.  A critical task is a task that
delaying its execution will prevent the completion of the whole application.
Castillo et al. \cite{7516037} propose a minimal extension to hardware design
that allows dynamic reconfiguration of multi-processors' per core
computational power.  By identifying critical tasks, the runtime is used to
guide the dynamic reconfiguration of hardware, so that tasks in the critical
path are given more computational power.
Myrmics~\cite{DBLP:journals/corr/LyberisPMN16} is a runtime system with task
dependency tracking, designed to scale on heterogeneous architectures.  Brumar
et al.~\cite{7967204} minimize redundant execution by exploiting repetitive
patterns in parallel workloads.  In their approach, a parallel task may not be
executed if a \emph{similar} one has already been executed before.  In this
case, the same result is reused.  They define a methodology for measuring task
\emph{similarity}.  This approach sacrifices result precision in order to improve
performance.  Vassiliadis et al. \cite{Vassiliadis:2015:PMR:2688500.2688546}
propose a task based model that aids the runtime system to identify less
\emph{significant} tasks and then decide whether it should execute them
accurately or approximately.  A less \emph{significant} task is defined as a task
that has small impact on the accuracy of the final result of the application.
They report reduction in energy consumption up to 83\%, when compared to a
fully accurate execution.  Jaulmes et al.  \cite{7832827}
expand the OmpSs programming model and its underlying runtime with error detection and
protection, for iterative solvers, in a transparent manner from the user's
perspective.  The effectiveness of automatic compared to manual vectorization in
task-parallel models is studied by Caminal et al. \cite{helenaArticle}.  

However, having a dedicated runtime aids in managing task creation,
synchronization, load balancing, data transfers, etc is not free.  Significant
overhead can be incurred by such a software system, which in many occasions can
limit the scalability of a parallel code or even perform worse than the
sequential version of the same code.  A typical way to deal with this overhead
is to avoid decomposing a workload into very small, fine-grained tasks or
processes, so that the actual work is always more than the execution time
required for the runtime system.  This can limit the available
parallelism significantly. Thus, the user often needs to find a
\emph{sweet spot} for the decomposition size of the workload.  Any software
approach however, is an order of magnitude slower than the equivalent hardware
implementation.

Future architectures should be designed in a way that they can use direct
information from the runtime system and also provide an infrastructure for
basic runtime functionalities (such as task creation and data tracking) to
eliminate any related overhead \cite{JSFI19,Casas2015}. Tan et al.
\cite{7967114} demonstrate the feasibility of the approach by implementing a
hardware accelerator on an FPGA, Picos++, which deals with tracking and
managing task data dependencies (e.g. OpenMP, OmpSs, IntelTBB).  This hardware
approach delivers 1.8$\times$ performance speedup and up to 40\% less power
consumption.  Etsion et al. \cite{etsion:micro2011} propose an abstraction to
out-of-order pipeline that operates at task granularity, instead of ILP.
Castillo et al. \cite{8327016} propose a hybrid software/hardware mechanism, where
data dependence tracking is offloaded to hardware, but task scheduling
is still managed at software level.  They report average speedup of 4.2\% 
over a hardware implemented runtime and require 7.3x less area, while 
the software scheduler is more flexible than a hardware implemented one.    

A significant body of work focuses on exploiting information available to the
runtime in order to guide and improve data management in memory. This is often
achieved with hardware extensions that allow the runtime system to communicate
with the memory subsystem.  Sanchez et al.
\cite{SanchezBarrera:2018:RDM:3205289.3205310} extract control and data
dependencies information from the runtime's task dependency graph in order to
reduce data transfers.  RADAR \cite{Manivannan.2016.HPCA} uses data
dependencies of tasks to track their memory footprint and find dead blocks in
last level cache memory.  Dead blocks can be then evicted from the cache.  Pan
et al. \cite{Pan.2015.SC} exploit the input annotations of tasks to identify
data blocks that will be reused in future tasks and use this information to guide
cache partitioning.  \'Alvarez et al.
\cite{Alvarez:2018:RMS:3205289.3205312} present a proposal for managing stacked
DRAM memories in HPC systems.  Stacked DRAM memories combine the benefits of
high-bandwidth DRAM with the large space of conventional off-chip memory.  In
their approach, the runtime is used to manage data transfers between memories
using idle workers, keeping all this functionality transparent to the user.
Papaefstathiou et al.~\cite{Papaefstathiou.2013.ICS} uses tasks' lifetime to
guide prefetching and data replacement in cache memories.  In the same spirit, Dimic et
al.~\cite{Dimic.2017.Europar} improve cache replacement policy to reduce the
miss ratio of last level caches.  \'Alvarez et al.~\cite{Alvarez.2015.PACT}
propose using data dependencies from the runtime system to manage scratchpad
memories.  Caheny et al.~\cite{Caheny:2016:RCC:2967938.2967962,
Caheny.2018.TPDS} propose adding another cache layer in the directory protocol
and exploit information available in the runtime system to reduce coherence
traffic in NUMA nodes.  The same authors \cite{Caheny:2018:RCC:3291656.3291703}
present a hardware/software hybrid system, which uses task-based and dataflow
model semantics to identify data that does not need memory coherence.  By
disabling coherence for such data, the system improves performance and energy
efficiency.  

Part of this thesis is inspired by the aforementioned runtime approaches and
exploits the underlying runtime to identify and mitigate the effects of
manufacturing variability (see Sections \ref{sec:process_variability} and
\ref{chap:power_aware_runtime}) present in multi-socket NUMA nodes.



%\section{Benchmarking in HPC}
%\label{sec:hpc_benchmarking}
%
Other studies exist that compare parallel programming models in the literature.  Although
these studies do not focus on task parallelism, they employ benchmarks and similar
methodology to evaluate their target models.  \cite{Coarfa:2005:EGA:1065944.1065950} study
and compare the performance of UPC and Co-array Fortran, two PGAS languages.  They use
select benchmarks from the NAS benchmark suite.
\cite{Appeltauer:2009:CCP:1562112.1562118} use microbenchmarks to measure and compare the
performance of 11 context-oriented languages.  Their study shows that they all often
manifest high overheads.


Although all the works we mention try to evaluate various programming models, in terms of
performance, and some times on usability and versatility, they are all limited to small
kernels or even just micro-benchmarks.  We find that this approach is not sufficient to
give us an insight on how a model will impact actual large-scale applications.
\cite{Karlin:2013:ETE:2510661.2511433} use a proxy application in their work to evaluate a
number of different programming models (OpenMP, MPI, MPI+OpenMP, CUDA, Chapel, Charm++,
Liszt, Loci).  Their approach however is limited to only one application.  Different
application domains can be very different, and may require different parallelization
techniques to get good scalability and performance.  A programming model could fail to
even provide a way to express a parallelization scheme, let alone deliver performance.  It
is important to have an in depth understanding of a models behavior and limitation in
order to make an educated decision whether research should direct its efforts to adopt
and further expand it. 


Pipeline parallelism has been the subject of study in some recent studies.  This
programming idiom is found often in streaming and server applications and goes far beyond
the HPC domain.  \cite{Lee:2013:OPP:2486159.2486174} propose an extension to the Cilk
model, for expressing pipeline parallelism on-the-fly, without constructing the pipeline
stages at their dependencies a priori.  It offers a performance comparison between the
proposed model, Pthreads and Thread Building Blocks (TTB) for three PARSEC benchmarks,
ferret, dedup and x264.

This trend of using microbenchmarks and kernel application is also followed when
evaluating other aspects of HPC, apart from parallel programming models, such as emerging
microarchitectures, novel load-balancing techniques and scheduling policies, etc.  The
SPEC CPU2006~\cite{Henning:2006:SCB:1186736.1186737} and SPEC
CPU2017~\cite{Bucek:2018:SCN:3185768.3185771} are benchmark suites designed to evaluate
processor architectures.  However, although the included workloads are fitting for
processor design evaluation, they are not representative of larger, more complex
applications that are run in today's large computer systems.  The PARSEC benchmark
suite~\cite{bienia2008} on the other hand is composed by applications from varying
computing domains, but are also common problems run on HPC systems.  Both SPEC and the
PARSEC however, are implemented using the most basic of parallel programming models, like
Pthreads.  Such programming models, although expressive enough to exploit the available
parallelism, offer little insight into how these applications interact with more
sophisticated programming models, which may have a dedicated runtime system to deal with
workload management and synchronization.  In this work we implement a variation of the
PARSEC benchmarks suite, the PARSECSs, using OMPSs/OpenMP 4.0 task directives and dataflow
relations.  Our implementation uses the most common features between contemporary
task-based models, so they can be easily ported.  Using task parallelism allows us to
implement more complex and efficient parallel programming paradigms, like pipelines.  In
this work, we will be using the PARSECSs to evaluate our runtime and job management
solutions to mitigating the manufacturing variability.
 

%\subsection{The \PARSEC{} Benchmark Suite}
%\label{sec:parsec}
%
\begin{table*}[!t]
	\centering
	\scriptsize
	\caption{\PARSEC{} Benchmark Suite}
	\begin{tabular}{|l|p{6cm}|p{3cm}|p{1cm}|}
	\hline
	\textbf{Benchmark} & \multicolumn{1}{|c|}{\textbf{Description}} & \multicolumn{1}{|c|}{\textbf{Native input}} & \multicolumn{1}{|c|}{\textbf{LOC}}\\
	\hline \hline
	blackscholes & Intel RMS benchmark. It calculates the prices for a portfolio of European options analytically with the Black-Scholes partial differential equation (PDE). & 10,000,000 options & 404 \\ \hline
	bodytrack & Computer vision application which tracks a 3D pose of a marker-less human body with multiple cameras through an image sequence. & 4 cameras, 261 frames, 4,000 particles, 5 annealing layers & 6,968\\ \hline
	canneal & Simulated cache-aware annealing to optimize routing cost of a chip design. & 2,500,000 elements, 6,000 temperature steps & 3,040\\ \hline
	dedup & Compresses a data stream with a combination of global compression and local compression in order to achieve high compression ratios. & 672 MB data & 3,401\\ \hline
	facesim & Intel RMS workload which takes a model of a human face and a time sequence of muscle activation and computes a visually realistic animation of the modeled face. & 100 frames, 372,126 tetrahedra & 34,134\\ \hline
	ferret & Content-based similarity search of feature-rich data such as audio, images, video, 3D shapes, etc. & 3,500 queries, 59,695 images database, find top 50 images & 10,552\\ \hline
	fluidanimate & Intel RMS application uses an extension of the Smoothed Particle Hydrodynamics (SPH) method to simulate an incompressible fluid for interactive animation purposes. & 500 frames, 500,000 particles & 2,348\\ \hline
	freqmine & Intel RMS application which employs an array-based version of the FP-growth (Frequent Pattern-growth) method for Frequent Itemset Mining (FIMI). & 250,000 HTML documents, minimum support 11,000 & 2,231\\ \hline
	raytrace & Intel RMS workload which renders an animated 3D scene. & 200 frames, 1,920$\times$1,080 pixels, 10 million polygons & 13,751\\ \hline
	streamcluster & Solves the online clustering problem. & 200,000 points per block, 5 block & 1,769\\ \hline
	swaptions & Intel RMS workload which uses the Heath-Jarrow-Morton (HJM) framework to price a portfolio of swaptions. & 128 swaptions, 1,000,000 simulations & 1,225\\ \hline
	vips & VASARI Image Processing System (VIPS), which includes fundamental image processing operations. & 18,000$\times$18,000 pixels & 127,957\\ \hline
	x264 & H.264/AVC (Advanced Video Coding) video encoder. & 512 frames, 1,920$\times$1,080 pixels & 29,329\\ \hline 
	\end{tabular}
	\label{tab:parsec}
	\vspace{1cm}
\end{table*}

With the prevalence of many-core processors and the increasing relevance of application domains that do not belong to the traditional HPC field, 
comes the need for programs 
representative of current and future parallel workloads. 
The \PARSEC{}~\cite{Bienia:PhD2011} features state-of-the art, 
computationally intensive algorithms and very diverse workloads from different areas of computing.
\PARSEC{} is comprised of 13 benchmark programs. 
The original suite makes use of the Pthreads parallelization model for all these benchmarks, 
except for \texttt{freqmine}, which is only available in OpenMP. 
The suite includes input sets for native machine execution, which are real input sets.
Table~\ref{tab:parsec} describes the different benchmarks included in the suite along with their respective native input and the lines of code 
(LOC) of each application.
We apply tasking parallelization strategies to 11 out of its 13 applications: \texttt{blackscholes}, \texttt{bodytrack}, \texttt{canneal}, \texttt{dedup}, \texttt{facesim}, \texttt{ferret}, 
\texttt{fluidanimate}, \texttt{freqmine}, \texttt{streamcluster} and \texttt{swaptions} and \texttt{x264}. 
We leave 2 applications out of this study: \texttt{raytrace} and \texttt{vips}.
\texttt{Vips} is a domain specific runtime system for image manipulation. 
Since vips is a runtime itself, it is not reasonable to implement it on top of another runtime system. 
Therefore we do not include this code in our evaluations. 
\texttt{Raytrace} code has the same extension as 
\texttt{ferret}, \texttt{facesim} and \texttt{bodytrack} and the same parallel model as \texttt{blackscholes}~\cite{Cook:2013:HEC:2508148.2485949}.
Therefore, since it does not offer any new insight, we do not consider the \texttt{Raytrace} code in this work.
 
We have a preliminary task-based implementation of the \texttt{x264} encoder, which scales up to 14x on a 16-core machine, the same as the Pthreads version.  
Since we just emulate the same parallel model as the original Pthreads version and obtain the same performance, we do not include this code in the results Section as it provides no insight.




\section{Managing HPC Clusters}
\label{sec:cluster_management}

In Section \ref{sec:cluster_design} we discuss how an HPC cluster is designed.  Typically,
an HPC cluster consists of multiple computing nodes, not necessarily of the same design
and computing capacity.  Building and maintaining an HPC cluster is a considerable
investment.  As such, the idle time of an HPC cluster should be minimized by any
institution or organization owning one, to make the most out of the machine.  It is not
uncommon to share an HPC cluster between multiple users, since few applications require
its full computing power.  Moreover, nodes may break but the system should still operate
with the rest, while a workload may require specific nodes, in an heterogeneous
environment.  The different workloads themselves may have different priorities and/or
dependencies between them.  All these aspects need to carefully managed to maximize a
cluster's job throughput.  All HPC cluster's today use dedicated software to manage their
workload and resources, such as SLURM \cite{slurm_02} and PBS
\cite{Feng:2007:PUP:1254882.1254906}.

In a typical cluster environment, the user writes a special script which runs the actual
parallel application and a few additional directives.  These directives describe
parameters related to the execution of the application, like the number of nodes and time
the application requires in order to run.  The script is then submitted to a queue, which
is maintained by the workload manager.  The submitted script and the application are
referred to as a job.  A user, or multiple users, may submit multiple jobs.  The workload
manager decides when and on which nodes a job will run.  The order that jobs are run is
dictated by the scheduling policy and the resources available.  As such, a workload
manager has two main functions: job scheduling and resource management.

\subsection{Job Scheduling}
Simple FIFO queues are not the most efficient way to run jobs on an HPC cluster. System
administrators may specify the scheduling parameters or implement their own policies to
match their needs.  Even in the simplest setups however, a FIFO queue is not sufficient.
Since jobs have varying execution time and may run on multiple or a single node.  Consider
the following example:  We have a cluster of 126 nodes, while job A needs to run for 30
minutes on 64 nodes and job B for 20 minutes on 100 nodes.  If job A is running, job B
must wait for A to complete, since there are not enough nodes.  Now consider that there
are more jobs after B.  Job C requires 16 nodes and 10 minutes to run.  In a simple FIFO
scheduling policy job C will need to wait for both jobs A and B to complete, which is
waiting for 50 minutes.  An alternative scenario, is for the workload manager to look
forward in the queue, since B cannot be scheduled, for jobs that require less nodes and
will complete within 30 minutes, so that running them before job B will not delay its
execution more than job A would.  This policy is called \emph{backfilling} and is the most
common one in contemporary workload managers \cite{10.1007/11407522_1}.  

\subsection{Resource Management} 
Resources are shared between jobs, so the workload manager needs to take care of which job
gets which resource.  It is not necessary that a node will run only a single job, so the
workload manager must take care of how it allocates cores and memory as well as the nodes
themselves.  Another important factor that a workload manager must consider is the
topology of the network and the distance of the nodes.  A multi-node job will perform
better if nodes are closer together, as synchronization and data movement would cost less.
In this work we argue that power, which is limited, should also be managed by the workload
manager.  In Section \ref{sec:power_aware_job_sched} we discuss the matter further and
provide related work which explores the potential benefits of power-aware scheduling
policies.  In Chapter \ref{chap:power_aware_job_scheduling}, we present our approach,
which employs power prediction models to drive scheduling decisions, considering the
available power on the cluster.


\section{Manufacturing Variability}
\label{sec:process_variability}

In the recent past, significant performance benefits were obtained by increasing number of
transistors on processor die, a phenomenon which became known as Moore's Law.  A different
interpretation of Moore's law is that of Dennard Scaling, which states that performance
per watt is doubling every two years, roughly.  To achieve this, engineers had to shrink
down transistors, which also means that the threshold voltage and current had to also be
scaled down.  As we reached a point of diminishing returns, neither Moore's nor Dennard's
observations hold true.  Shrinking transistors below a certain size leads to increased
sub-threshold conduction, leakage currents and heat dissipation.  It is not possible to
deal with the above issues and increase performance, without affecting a chips power
consumption and reliability \cite{Esmaeilzadeh:2013:PCM:2408776.2408797}.  Moreover,
decreasing transistor size makes lithography extremely challenging causing artifacts that
affect transistor parameters during the manufacturing process, such as distortions in film
thickness and channel length. The most important parameters affected are threshold voltage
($V_th$) and the effective gate length ($L_{eff}$), which can directly affect a transistor's
switching speed.  $V_th$ impacts the power leakage of transistors, while the switching
speed of transistors directly affects the chip's performance and power consumption
\cite{Borkar:2003:PVI:775832.775920,915379}.  

Since the manufacturing process cannot guarantee that transistors will operate at nominal
parameter values, processors of the same production line (stepping and firmware) manifest
variation in both their performance (frequency) and power consumption.  Most vendors use
frequency binning, meaning that processors with the same performance characteristics are
placed in the same group.  The same method is not employed however for binning together
processors with the same power consumption characteristics.  As such, modules processors
in current HPC systems are already inhomogeneous from the point of view of power.  Early
results on 64 processors have shown a 10\% power variation for identical workloads at
equivalent performance [41].  Despite the advances in fabrication process and power
gating, the impact of manufacturing variability is expected to worsen in future processor
generations \cite{1382598,Marathe:2017:ESP:3149412.3149421}. 

\subsection{Impact of Manufacturing Variability in HPC Systems}
In Section \ref{sec:cluster_homogeneity} we compared the homogeneous and heterogeneous
cluster designs and discussed the challenges users face when programming for the latter.
Typically HPC system operators obtain processors from the same bin, as by vendor
characterization, so that processors operate at he same frequencies.  This way,
homogeneity is guaranteed, at least for the units that it's intended, avoiding having to
deal with the unpredicted variability caused by the manufacturing process.  However, as
power is becoming a major concerned, an HPC systems can longer be perceived as homogeneous
in terms of power consumption.  Since vendors do not offer any classification of processor
variability, administrators of HPC systems ignore power consumption variability and treat
the system as homogeneous.  In this work, as with a number of recent related work found in
literature
\cite{Teodorescu:2008:VAS:1381306.1382152,Inadomi:2015:AMI:2807591.2807638,Gholkar:2016:PTH:2967938.2967961,Ellsworth:2015:DPS:2807591.2807643,Bailey:2015:FLP:2807591.2807637,Totoni:tech:2014},
it is demonstrated that it is important to consider power consumption variability in order
to improve the system's power efficiency and performance.  Furthermore, computer
architects have employed statistical model's \cite{4041872,1510283,article,4447311}  to
measure variation of processor's, since vendors do not release any relevant information on
their models.

It is not uncommon for HPC clusters to operate under system-wide power constrains.
However, since not all processors consume the same amount of power, it is not sufficient
to evenly power cap them.  Moreover, power capping a processor will translates power
consumption variability into frequency variability and thus, performance heterogeneity
\cite{Rountree2012}.  As such, a previously homogeneous system is now heterogeneous,
either from its power consumption or performance points of view, which in turn implies
that efficient use of the machine requires additional programming effort and/or software
support for mitigating its effects.  In this work we propose two different approaches for
providing software support: First at runtime level, where we propose methodology for
improving dynamic load balancing on power constraint sockets.  Secondly, we propose a new
analytical method for predicting power consumption variability, and then use the model to
guide job scheduling decisions.




   

 

\section{Variability-Aware Power Management in HPC Systems}
\label{sec:power_management}

As we head towards the exascale era, power is increasingly becoming a serious constraint.
full capacity.  According to a report from the US Department of Energy
\cite{ASCAC:tech:2014}, energy efficiency is considered to be among the top ten most
serious research challenges we face today, in order to achieve exascale computing
capacity.  The same report also points out that apart from the advances expected in
hardware, in order to reduce the power consumption of HPC systems, it is also necessary to
design software which is able to manage thousands of nodes in a power efficient manner.
Large HPC systems currently are expected to meet their power demands at all times, even
when all cores are operating at full capacity. This is rarely the case and poses a huge
strain on the power budget, provisioning the system for the worst case scenario, no matter
how unlikely it is.  A software solution could manage the underlying hardware and make
sure that such a scenario never takes place.  Moreover, modern clusters suffer from
manufacturing variability.  Any software solution should consider the heterogeneity in
power consumption in order to be effective.  It is a combination of energy efficient
hardware architectures and power-aware parallel runtimes and system software (e.g.
workload managers) that will make exascale computing feasible.       

\subsection{Software-aided Power Constraining and Management}
The most direct way of managing the power a processor consumes, is with DVFS.  Vendors 
typically allow their processors to operate at different frequency levels.  Lower frequencies
offer lower power consumption and DVFS can be set and managed by software.  Different cores
can also operate at different frequencies.
However, processor vendors, realizing the importance of managing power in future and current
HPC systems, are starting to offer greater flexibility by allowing users to set the exact 
power limit a processor can reach. 


The ability to set up power bounds in many-core systems is becoming a common feature.
For example, Intel introduced a set of machine-specific registers (MSRs)~\cite{libmsr} on their Sandy/Ivy Bridge processors to explicitly constrain on-chip power consumption. 
Although this seems as a straight-forward solution to managing power, system administrators must
consider manufacturing variability.  Processors do not consume the same power, even though they
are designed to do so.
In order to provide homogeneous performance,
chips of the same architecture must hide frequency variability, which can only be achieved via variations in their power consumption.
To abide to this user-set constraint, CPU cores resort to reducing their frequency. Under a power constraint however, different chips operate under different frequencies.
Since the release of commodity chips with such capabilities, several studies have shown the impact power capping can have. %have been made on power capped systems. 
In particular, work by Rountree et al.~\cite{Rountree2012} motivates the research presented in this 
thesis on how processor performance variability due to power capping can be addressed. 


In a power constrained environment where all chips need to operate under a certain power cap, this frequency variability can no longer be hidden~\cite{Rountree:2012:BDF:2357488.2357648}, leading to heterogeneous performance.
As a result, a theoretically homogeneous system turns into a heterogeneous one with performance variations of up to 64\%~\cite{Inadomi:2015:AMI:2807591.2807638}.  
While ignoring this manufacturing variability leads to performance 
and energy inefficiencies, 
there are opportunities for achieving improvements at the power budgeting or parallel runtime system levels when variability is properly managed~\cite{Chasapis:2016:RMM:2925426.2926279,Teodorescu:2008:VAS:1381306.1382152,Inadomi:2015:AMI:2807591.2807638,Gholkar:2016:PTH:2967938.2967961,Totoni:tech:2014}.
%Currently, state-of-the-art power-aware job scheduling approaches do not consider manufacturing variability %or process variation 
%to manage jobs dispatched across a parallel system.


Inadomi et al.~\cite{Inadomi:2015:AMI:2807591.2807638} also study the performance variability on a number production clusters and propose a variation-aware power budgeting framework.
Their approach requires specific single core executions for profiling the HPC applications plus a once-per-system profiling to build a reference table containing performance variability information for all nodes.
This table and the single core profiling is used to make decisions using a model. 
Compared to their method, our runtime approach does not require dedicated profiling runs or 
system wide reference tables containing performance variations.  Instead, we use profiling
information obtained at runtime to adjust power distribution and concurrency levels, 
which reduces the analysis costs and increases its benefits.

Bailey et al.~\cite{Bailey:2015:FLP:2807591.2807637} propose a linear programming formulation for MPI+OpenMP programs for maximizing performance under job-level power constraints. While this approach provides a good approximation of the upper bound of possible performance in dynamic runtime systems, the use of a linear programming solver is too slow to be practical for optimizing applications at runtime.
%, since the linear programming solver is too slow. 
The same group also introduced Conductor~\cite{conductor2015}, a dynamic runtime system that directs power to the critical path of the computation to minimize overall execution time under a power cap. Conductor, however, does not  
%While Conductor offers an online solution for finding optimal power configurations by exploiting the critical path of MPI applications, it does not 
deal with the hardware manufacturing variability we describe in this work.
As such, our approach is orthogonal and can be combined to maximize parallel applications performance.
%change the active core balance of the multi-core sockets involved in the parallel execution.
%Since such active core balance are an important factor to maximize performance under power constrained scenarios, our method extends the conductor approach and provides more benefits.
%at the runtime level.  
%However, it targets MPI applications, while our method focuses on OpenMP.

On a single node, Cochran et al.~\cite{Cochran:2011:PCA:2155620.2155641} classify the PARSEC benchmark suite applications for their power, temperature and performance characteristics. Using these results, they maximize performance while meeting power constraints by
%They propose a mechanism for maximizing performance and meeting power constrains 
using thread packing and DVFS. In contrast to our runtime approach, though, they rely on their priori characterization, while our approach can work without prior information.
%Optimal configurations are chosen using classification data they have collected a priori.
%our method 
%
%does not required any previoulsy collected data to maximize performance under a given power bound.

%On the level of system-wide power contraints, 
There is a significant body of work focused on job scheduling for power constrained systems.
Etinski et al.~\cite{5999809} propose an LP-based job scheduling policy; % is proposed in 
Sarood et al.~\cite{Sarood:2014:MTO:2683593.2683682} use performance modeling 
%is employed 
to make job scheduling decisions in power constraint system to improve job throughput;
and  Ellsworth et al.~\cite{Ellsworth:2015:PSD:2749246.2749277} discuss a dynamic job scheduling algorithm, which when running under a system-wide power limit, detects unused power and redistributes it to nodes that can make use of it.
%measures power 
%consumption of jobs 
%
%and based on a heuristic method reallocates power to sockets, to maximize performance.  

The impact of manufacturing quality on power consumption variability of processor chips has been studied in a significant number of works as well.
The power leakage of processors is directly connected to our work, since by setting a power limit on the socket, we impair its ability to adjust power consumption
to maintain the proper frequency level.
Davis et al.~\cite{61401478} study the effect of inter-node variability on power model characterization, in the context of homogeneous clusters.
Herbert et al.~\cite{Herbert:2012:EPV:2490159.2490164} show that exposing the power leakage variability of processors to the DVFS control algorithm to shift work to the less leaky processors, can reduce overall system power consumption.
Further, several projects study %Other works %have also taken into account 
the on-die power variation to improve DVFS scheduling~\cite{4919634,4798265,Teodorescu:2008:VAS:1381306.1382152}.
As an additional concern, modern processors require transistors to shrink to a level that introduces significant power and reliability variations among processors, a phenomena explored in detail by a variety of groups~\cite{Borkar:2003:PVI:775832.775920, 915379, 1046081}.
%Power and reliability variations in modern processors due to shrinking transitors sizes have been detailedly explored
Overall, most studies conclude that power variation is expected to become worse in the future~\cite{1382598,915379}, which will make the effects of power budgeting more apparent.  Thus, a 
variability-aware software solution is imperative in managing complex parallel applications and 
improving both performance and energy efficiency. 



\subsection{Power Prediction Models}

Knowing an applications power requirements can be invaluable information when making decisions
on when an where to run it in a large HPC cluster.  Relying on user supplied information is not
enough, since this information can be both difficult to obtain but also unreliable.
An alternative is to train analytical prediction models to get the information. 
Power prediction models have been extensively studied over the years.  In particular, models based on PMC
have been very successful in  predicting power consumption~\cite{Bircher:2012:CSP:2196827.2196987,Bertran:2012:SEC:2457472.2457499,Bertran:2010:DRP:1810085.1810108,Goel:2010:PSP:1909624.1909734,Isci:2003:RPM:956417.956567}.  
However, none of the models found in the literature consider manufacturing variability. 
In this work, we demonstrate that it is possible to make PMC prediction models aware of the 
manufacturing variability and accurately predict the socket and application specific power 
consumption, insteadof making the same prediction for all sockets.  
Our PMC-based model is based on the work of Bertran et al.~\cite{Bertran:2012:SEC:2457472.2457499,Bertran:2010:DRP:1810085.1810108}, which aims at providing insight into the way individual architectural components influence power consumption.
Our PMC-based model extends Bertran's model to account for manufacturing variability in terms of power consumption.
The original model requires carefully crafting micro-benchmarks that isolate activity per architectural component, in order to train it.  We show that comparable results can be obtained 
by training the model with a small set of kernel applications and a microbenchmark that stresses the memory unit.
The benefit of using the suggested set of applications for training the model, is that it is a more
portable solution, compared to Beltran's original set of microbenchmarks.  In Beltran's work, 
the microbenchmarks need to be designed so that only certain architectural components are stessed
by each microbenchmark.  This is achieved by carefully implementing an assembly code and considering
all the architectural details of the underlying machine.  Although our approach will suffer a small
penalty in precision, there is not need of modification when run on a different machine.
%, which makes their approach architectural dependent.
Our approach intends to predict power consumption variability while remaining easy to 
deploy on any system.
The implementation, application and evaluation of our proposed analytical
analytical prediction model is discussed in Chapter \ref{chap:power_aware_job_scheduling}. 


\subsection{Power Aware System-Wide Job Scheduling}
\label{sec:power_aware_job_sched}

Handling power has become an important factor when managing and designing HPC
systems.  However, contemporary workload managers deployed on such systems (e.g. SLURM \cite{slurm_02})
 do not consider power as a resource and do not
manage their workloads in an energy efficient way.  Yet, researchers have
identified the need to make workload management software power-aware and have
already published various experimental approaches 
\cite{Sarood:2014:MTO:2683593.2683682,932708,Etinski2010,Etinski2012615,10.1007/978-3-319-07518-1_25,1559953,8081827,KHEMKA201514,8081827,LEAL201633,Patki:2015:PRM:2749246.2749262}.

A survey on the techniques developed in nine of the TOP500 HPC centers for improving energy
efficient is presented by Maiterth et al. \cite{8425478}.  They identify several emerging
techniques, some with common characteristics.  Over-provisioning
\cite{Sarood:2014:MTO:2683593.2683682} considers building a system where it is not
possible to run all the nodes at full capacity.  Instead, the system operates under a
certain power budget and dynamically distributes the available power among nodes.  Nodes can
operate under different power caps.  For example, a few nodes may operate at full capacity, while
the rest are disabled or constrained.  Other
approaches \cite{932708,Patki:2015:PRM:2749246.2749262} take advantage of applications that
can be considered \emph{moldable}, meaning that these applications can run at different
configurations (e.g. number of threads). 

A significant body of work examines approaches on
how to optimally use DVFS or hardware imposed power constrains (e.g. RAPL) in order to
save energy but also optimize performance~\cite{Etinski2010,Etinski2012615,10.1007/978-3-319-07518-1_25,1559953,8081827,Patki:2015:PRM:2749246.2749262}.
A different approach is also identified, where instead of using hardware imposed power
caps, energy efficiency is achieved only by job scheduling~\cite{KHEMKA201514,8081827,LEAL201633}.  
Manufacturing variability is also considered in
some studies, which exploit the variance in power and performance among nodes to improve
energy efficiency \cite{Patki:2015:PRM:2749246.2749262,ShoukourianPhD}.  Although the
identified techniques are not used in production in any of the HPC centers, they provide some 
insight on future trends in energy efficient HPC computing.

Etinksi et al. \cite{Etinski2010} present a practical approach to apply DVFS on an HPC
cluster, exploiting periods of low activity.  With DVFS, scaling down the frequency to 
save power causes significant performance degradation.  Their approach manages to reduce
the negative impact DVFS by applying it when overall activity is low on the cluster. 
Moreover, Etinksi et al. \cite{Etinski2012615} present
a power-aware job scheduling policy, MaxJobPerf.  Their policy considers two types of
resource that need to be allocated to new jobs, processors and power.  To
decide which job should be scheduled next and how power should be distributed
among jobs, they use integer linear programming.
Sarood et al.~\cite{Sarood:2014:MTO:2683593.2683682} use performance modeling
to increase job throughput in power constrained systems.  A power management of
overprovisioned systems has also been studied by Patki et
al.~\cite{patki:2013:eho:2464996.2465009,7515666}. 
Unlike our work, all sockets in a cluster are viewed as homogeneous in terms of power 
consumption, which can lead to suboptimal scheduling decisions.

More recent work identifies the need to consider manufacturing variability when making
scheduling decisions or managing a system's power budget
~\cite{Inadomi:2015:AMI:2807591.2807638,Gholkar:2016:PTH:2967938.2967961,Ellsworth:2015:DPS:2807591.2807643,Bailey:2015:FLP:2807591.2807637,Teodorescu:2008:VAS:1381306.1382152,Totoni:tech:2014}.
Inadomi et al.~\cite{Inadomi:2015:AMI:2807591.2807638} extensively study the impact of
manufacturing variability on a number of production clusters and propose a variation-aware
power budgeting framework.  They introduce variability to their prediction model by statically 
measuring power variability on
each socket and then apply it to their original, variability agnostic, predictions.  However,
they base their approach on the assumption that variability is application independent. 
%In
%contrast to our work, their prediction model is used to guide work balancing within an MPI
%application and not system wide job scheduling.  Unlike our \textit{Optimized PMC} model,
%Inadomi's and our \textit{PR} models assume that variability is application independent,
%which is not correct in general.  As the effects of manufacturing variability are expected
%to increase in future CPUs \cite{Marathe:2017:ESP:3149412.3149421}, a more robust model is
%needed, such as our \textit{Optimized PMC} model.  
Teodorescu et al.~\cite{Teodorescu:2008:VAS:1381306.1382152} study the impact
of manufacturing variability and propose a linear programming algorithm to find
the best parameters for power budgeting with DVFS.  Ellsworth et
al.~\cite{Ellsworth:2015:DPS:2807591.2807643} propose a power distribution
framework that optimizes an HPC cluster's power consumption under a certain
system wide power budget.  In contrast to our work, jobs are scheduled without
considering power consumption, but power is redistributed, favoring more power
intensive jobs.  A two level solution for overprovisioned clusters is presented
by Gholkar et al.~\cite{Gholkar:2016:PTH:2967938.2967961}, where a job
scheduler is used at system level to allocate nodes and distribute power.  The
job scheduler predicts the total energy consumption in order to make a
scheduling decision.  Individual sockets may run under power constrains, in
which case, a second runtime scheduler decides the optimal configuration of
active processors and the power distribution among them in order to mitigate
the power variability.  Adagio \cite{rountree2009} detects the critical path of
MPI applications and uses DVFS to reduce power consumption of non-critical
pieces of work, hiding performance variability from the scheduler.

In this thesis we present two job scheduling policies that consider
manufacturing variability to optimize performance and energy efficiency, in
power constrained clusters.  We use two different analytical models to predict
power consumption and manufacturing variability, in order to guide scheduling
decisions.  The first model is similar to
\cite{Inadomi:2015:AMI:2807591.2807638}, making the same assumption that
variability is application dependent, but used in a different context.  The
second model offers a more robust approach, eliminating the aforementioned
assumption.  In contrast to hardware imposed power caps, our approach aims to
maintain power consumption below a certain budget, only by optimizing job
scheduling.  This way we avoid the implications of degraded and varying
performance, of power constrained sockets.

%Our job scheduling policies could also be coupled with energy saving runtimes, to further reduce power consumption.  However, any technique that throttles frequency or power, such as VDD, 
%will expose the performance variability of the sockets.  Our models would need to be extended to consider different frequencies during training.   
%A more transparent approach is to couple it with a runtime like Adagio \cite{rountree2009}.
%It detects the critical path of MPI applications and uses DVFS to reduce power consumption of non-critical pieces of work, hiding performance variability from the scheduler.



